% 请在下方的大括号相应位置填写正确的节标题和标签,以及作者姓名
\section{sed文本替换}\label{sec:sed文本替换}
\sectionAuthor{Jiaqi Z.}

% 请在下方的item内填写本节知识点
\begin{Abstract}
    \item 如何使用\code{sed}命令对文本文件进行替换
    \item 如何使用\code{sed}查看文件内容
    \item 如何使用\code{sed}添加(删除)文本文件的内容
\end{Abstract}

% 请在正文相应位置填写正确的小节标题(或小小节标题),同时将标签的“节标题”和“小节标题”改为实际内容

在\ref{sec:查找与替换}当中,已经介绍了如何使用vim进行文本文件的替换。与vi里面的\code{s}命令类似,在命令行也有类似的方式可以对文本文件进行替换(甚至更复杂的操作)。那就是使用\keyword{sed}命令。

\subsection{使用\code{sed}显示}\label{subsec:sed文本替换-使用sed显示}

\code{sed}的语法格式是:\code{sed [选项] "<动作>" [文件路径]}。其中,常见的选项包括

\begin{itemize}
    \item \keywordin{sed}{-n} 表示\emph{只有处理的文本}显示在屏幕上(默认是全部文本)
    \item \keywordin{sed}{-i} 表示\emph{直接修改文件内容}。
\end{itemize}

\begin{attention}
    在本节,我们所处理的文件都会使用\code{-i}选项直接对文件进行处理。但这一行为是危险的,尤其是对于不确定的修改操作,由于它是对原文件进行修改,因此需要提前备份。

    在下一节将介绍管道运算符和重定向运算符,将会帮助你避免这一问题。
\end{attention}

\begin{extend}
    如果你仔细观察上面的语法格式,需要特别注意的是里面的\code{[文件路径]}是“可选”的,也就是说,\emph{对于\code{sed}命令而言,可以不提供文件路径}。这件事可能会比较“难以置信”,毕竟如果没有文件,怎样处理文件呢?

    事实上,\code{sed}命令是一个\emph{管道命令},简单来说,它是可以读取并处理终端输出的内容。除此之外,像前面的\code{grep}也是类似的(虽然我们前面并没有特别强调这一点)。

    在下一节,我们将详细讨论管道命令的内容。
\end{extend}

让我们先来看一个最简单的操作——\emph{打印},即试着将特定几行的内容打印在屏幕上。其在“动作”部分的写法是\code{<开始行号>,<结束行号>p}。例如,我们希望将\code{POSCAR}的第3-5行输出,可以使用命令\code{sed -n "3,5p" POSCAR}表示输出第3-5行内容。

\begin{extend}
    如果你此时忘记了\code{-n}选项的话,命令将会输出所有的\code{POSCAR}文件,但对应行号的内容会重复输出。(可以试试)

    另外,如果希望对最后一行进行操作,需要使用\code{\$}符号表示\emph{最后一行},但此时\emph{需要将外面的双引号变为单引号}。通常情况下,双引号和单引号是等价的,但在涉及到\code{\$}符号时,使用双引号则表示将\code{\$}和后面的符号(例如\code{\$p})解析成\emph{变量p},而使用单引号则表明这一符号就是符号本身。变量的作用,将在后面循环和批量处理中体现,在此不做讨论。
\end{extend}

\subsection{使用\code{sed}添加和删除}\label{subsec:sed文本替换-使用sed添加和删除}

使用\code{sed}命令删除某一行的方法是使用\code{d},例如,如果希望删除\code{POSCAR}文件的第3-5行,类似于上面的显示,使用方法是\code{sed -n "3,5d" POSCAR}

\begin{attention}
    这一操作是对文件内容进行删除,因此你应该谨慎使用\code{-i}选项(会直接将原文件对应内容删除),在下一节的重定向运算符将会提供灵活的解决方法。目前的一个方法是\emph{创建一个备份文件}。
\end{attention}

对于添加某行内容,可以使用\code{a}或\code{i},前者表示\emph{新增(在下一行)},后者表示\emph{插入(在上一行)}。具体的,其命令为:\code{sed [选项] "行号a(i) 添加字符串" [文件名]}例如,希望在\code{POSCAR}的第一行后面添加一个字符串“1.0”,可以使用\code{sed -n "1a 1.0" POSCAR}。

那么,如果希望在第一行添加内容,应当怎么办呢?

\answer{\code{sed -n "1i 1.0" POSCAR}即可。}

\subsection{使用\code{sed}替换}\label{subsec:sed文本替换-使用sed替换}

使用\code{sed}替换有两种模式,分别是\emph{整行替换}和\emph{字符串替换}。对于整行替换,命令为\code{sed [选项] "行号c 替换目标字符串" [文件名]}。例如,若希望将\code{POSCAR}的第2行替换为“0.8”,则可以使用命令:\code{sed -n "2c 0.8" POSCAR}。

相比于整行替换,字符串替换可能更为强大。与vim的替换类似,在\code{sed}当中替换字符串的命令是\code{sed [选项] "s/要被替换的字符串/新字符串/g"}。例如,希望在\code{INCAR}当中的\code{ENCUT=400}替换为\code{ENCUT=600},可以使用命令\code{sed "s/ENCUT=400/ENCUT=600/g" INCAR}。

对于字符串替换而言,更重要的是\emph{支持正则表达式}(关于正则表达式的内容详见\ref{sec:初窥正则表达式}一节),例如,在上面的例子中,我们希望将\code{ENCUT=400}或\code{ENCUT=600}全部替换为\code{ENCUT=800},则可以使用\code{sed "s/ENCUT=$\backslash$ d$\backslash$ d$\backslash$ d/ENCUT=800/g" INCAR}

\begin{attention}
    同样,由于这是对内容的删除(更改),因此需要提前备份。除非有足够的把握,否则不要使用\code{-i}选项。
\end{attention}


\subsection{错误处理}\label{subsec:sed文本替换-错误处理}
% 请在本节列出可能遇见的错误与解决方法

\subsubsection{sed: -e expression \#1, char 2: invalid usage of line address 0}

在使用行号时,第一行是1而不是0. 通常来说,当你试图在第1行插入数据时,应当使用\code{sed -n "1i 字符串" [文件名]},其中\code{i}表示在第1行前面插入数据(即在第一行写入字符串)