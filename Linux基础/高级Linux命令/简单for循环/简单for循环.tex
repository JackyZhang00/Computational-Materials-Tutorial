% 请在下方的大括号相应位置填写正确的节标题和标签,以及作者姓名
\section{简单for循环}\label{sec:简单for循环}
\sectionAuthor{Jiaqi Z.}

\begin{Abstract}
    \item 命令行\code{for}循环的基本结构
    \item 如何使用\code{for}循环批量处理任务
    \item \code{\$OLDPWD}变量的使用
\end{Abstract}

在这一章的最后,让我们看一下关于命令行的最后一个话题——\emph{批量处理}。其中,批量处理的一个基本方法是使用\keyword{for}循环语句。

\begin{attention}
    更复杂的批量处理可能需要配合下一章介绍的bash批处理脚本,那里面会有进一步复杂的如条件判断等。一般来说,更复杂的批量处理会放到脚本中执行。在本节,我们仅仅讨论一些能使用\code{for}循环简单处理的任务。
\end{attention}

\begin{extend}
    虽然说是“最后一个话题”,但实际上关于命令行的使用远不止此。只不过目前教程(或者说科研过程中)可能涉猎到的也就这些。一些更复杂的,或者更细节的使用,例如系统本身操作等,我们并没有对此进行讲解。对于需要的人来说(例如服务器维护相关人员),请查阅更专业的书籍或相关手册了解更多如系统目录架构,以及服务器维护,root权限等相关内容。\footnote{我们不讲解还有一个原因:担心一般使用时出现错误导致服务器崩溃等异常情况出现。}
    
    对于一些后续可能会用到的命令,会有新的补充,请继续关注仓库内版本更新,或者网页端动态更新。
\end{extend}

\subsection{\code{for}语句基本结构}\label{subsec:简单for循环-for语句基本结构}

在Linux当中,\code{for}语句的基本结构为:\code{for [变量名] in [列表范围]; do [要执行的语句]; done}。其中,\code{[列表范围]}可以使用大括号将其依次列出,例如\code{\{a,b,c,d,e\}}表示\emph{遍历这五个字母}。同时,在\code{do}后面需要是按顺序执行的语句,其中每一个语句后面以分号(\code{;})结尾。

\begin{attention}
    在Linux当中,分号表示一个命令的结束,对于最后一个语句可以不加分号(此时回车表示结束)。例如,希望一次性输出\code{POSCAR}和\code{CONTCAR}文件的最后几行,可以一次性运行两句命令\code{tail POSCAR; tail CONTCAR},其中分号表示这是两个语句。

    在\code{for}循环中,你也可以这样理解分号,其中\code{done}前面只需要有一个分号(表示语句与\code{done}的分割)。换言之,\emph{不可能有两个分号相邻}。
\end{attention}

\subsection{关于变量}\label{subsec:简单for循环-关于变量}

如果只是重复执行几个没有丝毫变化的语句,似乎有点“无趣”。但我们如何在命令中加入一些可以变化的东西呢?那就是\emph{变量}。在Linux命令行当中,变量是以\code{\$}符号表示的。例如,\code{\$i}表示变量\code{i}。因此,如果我们希望一次性创建a到e五个目录,则可以使用下面的语句:\code{for i in {a,b,c,d,e}; do mkdir \$i; done}。其中,\code{for i in {a,b,c,d,e}}表示\emph{遍历后面的列表,并分别将变量\code{i}赋值为列表中的元素};后面的语句表示执行\code{mkdir}命令,但其中的参数为变量\code{i},例如,第一次时执行的为\code{mkdir a},第二次就为\code{mkdir b},以此类推。

\begin{extend}
    在所有变量中,有一些变量是比较特殊的,它们具有特定的含义。比较常见的如\code{\$PWD}表示\emph{当前所在的目录路径};\code{\$OLDPWD}表示上一个所在的目录路径。

    在一般的命令行中,你也可以在必要的时候使用这些命令(哪怕不是在循环中)。例如,\code{echo \$PWD}表示打印当前路径\footnote{\keyword{echo}表示输出显示某一个内容,其后面的参数表示输出的内容。}。(事实上,你也可以直接使用\keyword{pwd}命令快速实现这一功能)

    特别注意的是,\code{\$OLDPWD}表示上一个所在的目录路径,而不是上一级目录。例如,原先在\code{1/a/}目录下,当你使用\code{cd 1/b/}切换到\code{1/b/}目录时,变量\code{\$OLDPWD}表示的是\code{1/a/}而不是它的上一级目录\code{1/}。使用\code{cd \$OLDPWD}可以帮助你快速切换到上一个操作的目录,但还有一个更简单的方法是使用\code{cd -},这二者是等价的。
\end{extend}

\begin{attention}
    就目前在写作的过程中,我还没有具体了解到如何在\code{for}循环中使用两个变量。类似于\code{for i,j in {a,b,c},{1,2,3}}这种写法是\emph{不可行的}。如果你了解到了具体实现这一效果的方法(在命令行下),请通过前面的联系方式联系我。
\end{attention}

\subsection{\code{seq}序列}\label{subsec:简单for循环-seq序列}

在前面使用\code{for}循环时,我们需要把遍历元素全部列举出来。通常这种方法适合于一些没有特定模式的序列,且数量较少。对于数量较多,或者我们明确知道其规律(通常是一些数字序列),一般会选择使用序列的方式让命令行自动生成。在Linux当中,最普遍的\emph{数字序列}生成的方法是使用\keyword{seq}命令,其形式为\code{seq <起始数值> <步长> 终止数值}。当只有一个参数时,起始数值和步长默认为1;两个参数则分别表示起始数值和终止数值(此时步长默认为1)。

当希望在\code{for}循环中使用序列时,需要将\code{seq}命令(包括后面的参数)使用反引号(\code{`})括起来。例如,\code{for i in `seq 1 2 10`}表示对序列\code{{1,3,5,7,9}}进行遍历。

\begin{attention}
    正如你所见,\code{seq 1 2 10}表示序列\code{{1,3,5,7,9}}而不包括11,这是因为11超过了终止数值10.而对于\code{seq 1 2 11}则包括11.这一点可能与其他编程语言如Python不同,在Python当中,\code{range(1,11,2)}不包括最后的11.

    \code{seq}命令可以使用负步长表示递减序列。例如,\code{seq 5 -1 1}表示序列\code{{5,4,3,2,1}}。

    同时,\code{seq}也可以使用浮点数,例如,\code{seq 0.1 0.1 0.5}表示序列0.1到0.5,间隔0.1.
\end{attention}

除此之外,对于一些整数的序列,也可以使用更简单的方式,其形式为\code{{起始数值..终止数值..<步长>}}。例如,\code{for i in {1..10}}表示对1到10进行遍历;而\code{for i in {1..5..2}}则对\code{{1,3,5}}遍历。

与\code{seq}不同的是,使用\code{..}的表示方法还可以对字母序列进行表示,例如,\code{{a..z}}表示所有小写字母,同理\code{{A..Z}}表示所有大写字母。

\begin{attention}
    同样的,你也可以使用\code{A..z}对大写字母和小写字母进行表示,但此时\emph{还包括一些特殊字符如[,]等},它们是在ASCII码介于字母中间的部分(91到96)。

    特别注意的一点是,使用\code{..}的方法不能对浮点数进行操作。当然,对于负步长仍然可用,例如,\code{5..1..-1}表示序列\code{{5,4,3,2,1}}。但对于这一方法,更特别的是你可以省略后面的步长(加上也不错),Linux会自动做降序。更甚者,你可以使用\code{{10..2..2}}的方法来生成\code{{10,8,6,4,2}}这一序列(此时步长完全是错误的)。
\end{attention}

最后再来总结一下,对于整数序列,无论使用哪种方式都能得到;而对于小数(浮点数)序列,只能使用\code{seq}方法;对于字母序列则只能使用\code{..}方法生成。


\subsection{一些\code{for}循环使用例}\label{subsec:简单for循环-一些for循环使用例}

下面,讨论一些\code{for}循环的使用方式,仅供启发用(实际使用时可能需要根据实际情况进行个性化调整)。

首先,如果我们希望生成序列为{200,400,600,800}的目录,则可以使用命令\code{for i in {200..800..200}; do mkdir \$i; done}。正如你所见的那样,这一命令表示对于这样一个序列进行遍历,执行\code{mkdir}命令,其中参数为变量\code{i},依次为200,400,600,800(前面的序列)。

下面,假设我们在当前目录下有一个文件\code{INCAR},希望将其拷贝至这里面每一个目录下,可以使用命令\code{for i in {200..800..200}; do cp INCAR \$i; done}。其中表示对序列的每一个变量\code{i},执行\code{cp}命令,其中参数(拷贝目标路径)分别设置为序列里的元素。

最后,当我们希望将每个目录里面的\code{INCAR}里面的\code{ENCUT = 200}分别改为\code{ENCUT = [目录名]},可以使用下面的方法:\code{for i in `seq 200 200 800`; do cd \$i; sed -i "s/200/\$i/g" INCAR; cd \$OLDPWD; done}。这表示对前面的序列\footnote{这里我们使用\code{seq}方式生成,仅是为了多一种展示,也可以使用\code{200..800..200},它们是一样的。},依次执行下面的操作:

\begin{enumerate}
    \item 进入其目录;
    \item 使用\code{sed}命令修改(使用方法详见\ref{sec:sed文本替换}一节),其目标文本为变量\code{\$i},也就是目录名;
    \item 返回到之前的目录(以便于能够继续遍历其他元素,进入对应目录)
\end{enumerate}

\begin{extend}
    你可以试一试,如果没有后面的\code{cd \$OLDPWD},会发生什么?稍加分析可以看出,当进入\code{200}目录时,修改完毕之后,会进入下一次循环遍历,此时尝试执行\code{cd 400}。但当前目录下并没有该目录(200和400是并列关系,不是父子目录关系),因此程序会报错。具体的报错可以看下面的\ref{subsec:简单for循环-错误处理}一部分。
\end{extend}


\subsection{错误处理}\label{subsec:简单for循环-错误处理}
% 请在本节列出可能遇见的错误与解决方法

\subsubsection{-bash: cd: <目录名>: No such file or directory}

这就是前面所说的忘记\code{cd \$OLDPWD}的错误信息,此时会发现,没有目录,怎么办?只能报错了(如果你足够“机敏”,也许你会联想到\ref{sec:目录操作}一节。没错,在那里的\ref{subsec:目录操作-错误处理}当中,也有这一错误。实际上,二者的本质是相同的。

\subsubsection{seq: invalid floating point argument: <字符>}

这是因为你在使用\code{seq}时尝试生成非数值(整数或浮点数)序列。例如,你也许本意是生成字母a到z的序列,但使用\code{seq a z}则会产生上面的错误。正确方法是使用\code{{a..z}}

\subsubsection{使用\code{..}的方法生成浮点数,结果错误}

使用\code{..}不能生成浮点数。因此,如果你希望生成浮点数,一种方法是使用\code{seq},当然,还有一种“投机取巧”的方法,即尝试将\emph{浮点数用整数表示}。例如,如果希望生成\code{{0.1,0.2,0.3,0.4,0.5}}的目录,除了使用\code{for i in `seq 0.1 0.1 0.5`; do mkdir \$i; done}外,还可以使用\code{for i in {1..5}; do mkdir 0.\$i; done}。其中\code{0.\$i}的\code{\$i}需要替换为遍历的序列(整数),也就是\code{{0.1,0.2,0.3,0.4,0.5}}。