% 请在下方的大括号相应位置填写正确的节标题和标签,以及作者姓名
\section{管道与重定向}\label{sec:管道与重定向}
\sectionAuthor{Jiaqi Z.}

% 请在下方的item内填写本节知识点
\begin{Abstract}
    \item 管道运算符、如何使用管道运算符连接多个命令
    \item 使用重定向写入(读取)文本文件
\end{Abstract}

% 请在正文相应位置填写正确的小节标题(或小小节标题),同时将标签的“节标题”和“小节标题”改为实际内容

\subsection{管道}\label{subsec:管道与重定向-管道}

在\ref{sec:sed文本替换}一节当中,我们讨论了如何使用\code{sed}命令对文本文件进行编辑。当时我们提到,\code{sed}命令是一个管道命令,可以读取终端输出的内容。除此之外,在很多时候我们希望对某一文件进行多次操作,例如,提取某一文件的前10行,并将其中的“C”改成“B”,然后再写入新的文件当中。

像上面这种输入和输出层层传递的(一个命令的输出作为另一个命令的输入),就可以使用这一节的“管道”命令。在Linux当中,管道运算符是\code{|}(通常是Shift+Backspace下面的那个键),它的作用是\emph{把前面命令的输出作为下一个命令的参数输入}。例如,我们希望将\code{POSCAR}输出,可以使用\code{cat POSCAR},若此时又想取出前5行,则可以使用\code{cat POSCAR | head -n 5}

\begin{attention}
    也许使用\code{head -n 5 POSCAR}一样可以解决上述问题,但使用管道更具有“可扩展性”。例如,\code{cat}命令本意是将多个文件连接起来,若希望将连接之后的文件读取前5行,则使用管道运算符是最简单的方法之一。
\end{attention}

上面的过程,可以看作是将\code{cat}命令输出结果作为\code{head}命令的参数输入,之后运行\code{head}命令并输出(至标准输出),进一步,如果我们希望将其中的“C”全部替换成“B”,则需要借助于\code{sed}命令,表示为:\code{cat POSCAR | head -n 5 | sed "s/C/B/g"}

\begin{attention}
    此时使用\code{sed}命令由于没有添加\code{-i}选项,因此结果也仅仅在标准输出当中进行输出,源文件并没有修改。
\end{attention}

类似地,如果我们希望得到\code{OUTCAR}最后一个包含“without”字符串的行,在使用管道之前是“几乎不可能”的,而在利用管道时,便可以使用命令:\code{grep without OUTCAR | tail -n 1}得到结果\footnote{这一过程实际上是在VASP当中得到最后收敛能量的过程。}。

\subsection{输出重定向与输入重定向}\label{subsec:管道与重定向-输出重定向与输入重定向}

在之前,我们得到的输出结果仅仅是在屏幕上输出(也被称为“标准输出”),但必要的时候,我们也希望将结果保存至本地,以便后续处理(无论是进一步使用程序语言读取并处理,还是过一段时间再看)。正因如此,寻找一种方法将输出结果保存就十分重要。

在Linux当中,保存终端输出的本质就是“将输出\emph{重定向}至文件”,其运算符是\code{>}或\code{>>},后面需要有一个文件名路径表示\emph{希望写入的文件}。其中,前者(\code{>})表示\emph{创建},当文件存在时则会覆盖;后者(\code{>>})表示\emph{追加},当文件不存在时新建,存在时则会在后面追加新的内容。

有了这一方法,我们终于可以解决\ref{sec:sed文本替换}一节所遗留的关键问题:如何将编辑后的文件保存至新的文件?答案就是\emph{使用重定向运算符}。例如,我们希望将\code{INCAR}文件里面的\code{ENCUT=400}改为\code{ENCUT=600}并保存至\code{INCAR2},则可以使用命令:\code{sed "s/ENCUT=400/ENCUT=600/g" INCAR > INCAR2}

再一个例子,如果希望将\code{Si/POTCAR}和\code{O/POTCAR}合并至一个新的\code{POTCAR}当中,应当怎样写呢?

\answer{\code{cat Si/POTCAR O/POTCAR > POTCAR}}

\begin{extend}
    请注意运算符是\code{>}而不是\code{<},前者表示“输出重定向”,而后者表示“输入重定向”,即将文件的内容作为命令的输入。例如,\code{cat < POSCAR}与\code{cat POSCAR}等价。

    二者的方向虽然容易混淆,但似乎可以从箭头的方向看出一点规律——\code{>}表示将命令的内容“输出至”文件中,而\code{<}表示将文件“输入至”命令中。
\end{extend}

在必要的时候,我们当然也可以将输入和输出同时重定向,例如,\code{cat < POSCAR > POSCAR2}也是可行的(将\code{POSCAR}重定向输入至\code{cat}命令,并将命令输出结果重定向至\code{POSCAR2}输出)

\begin{extend}
    除了“标准输入”和“标准输出”之外,Linux还有一个“标准错误输出”,用来输出命令运行报错的结果。例如,当我们希望删除一个名为\code{POSCUT}的文件时(该文件并不存在),使用\code{rm POSCUT}会报错(这一点在\ref{sec:文件操作}已经详细讨论过了)。如果试图将这一输出重定向,例如,\code{rm POSCUT > output},效果是一样的。但如果使用\code{rm POSCUT 2> output}呢?你会发现,终端没有报错了,而将报错输出至文件\code{outcar}当中了。这其中,\code{2>}就表示将“标准错误输出”重定向至后面的文件。

    在这一基础上,稍微扩展一下。在后面的VASP教程中,我们将看到提交脚本中有\code{mpirun vasp\_std > vasp.out 2>vasp.err}这一行\footnote{不同课题组可能有所不同,这里仅仅作为例子。}。暂且忽略掉前面的\code{mpirun}(表示分布式计算系统下并行运行任务),可以发现,这一行的作用就是运行\code{vasp\_std},并将输出结果输出至\code{vasp.out},而将错误信息输出至\code{vasp.err}。

    另外,我们这里重定向的文件并不一定是文本文件。如果你回顾一下\ref{subsec:查看文件-Linux文件类型}一节,可能会发现里面有一个“设备文件”,它也是可以作为重定向输入输出的一部分。例如,\code{rm POSCUT 2> /dev/null}则表示将输出报错信息重定向至\code{/dev/null}设备(这是一个“空设备”,用于消除所输出的内容)。运行这一命令,你将不会得到任何输出结果(输出被消除了)
\end{extend}


\subsection{错误处理}\label{subsec:管道与重定向-错误处理}
% 请在本节列出可能遇见的错误与解决方法

\subsubsection{使用\code{cat << POSCAR}没有反应}

如果你确实希望使用输入重定向,应当注意是\code{<}(一个)而不是\code{<<}(两个)。后者在Linux中通常用于终端交互中,例如,上面的命令表示将\code{POSCAR}之间的内容作为输入。一个演示例子为:

\begin{lstlisting}[language=bash]
$ cat << POSCAR
> hello
> world
> POSCAR
hello
world
\end{lstlisting}

可以看到,它把“POSCAR”之间的内容传入了\code{cat}命令(输出)


