% 请在下方的大括号相应位置填写正确的节标题和标签,以及作者姓名
\section{变量}\label{sec:变量}
\sectionAuthor{Jiaqi Z.}

% 请在下方的item内填写本节知识点
\begin{Abstract}
    \item 如何在脚本中定义变量
    \item 如何输出变量
    \item 如何对变量进行简单运算
\end{Abstract}

% 请在正文相应位置填写正确的小节标题(或小小节标题),同时将标签的“节标题”和“小节标题”改为实际内容

如果所有脚本都只能按照固定的内容运行,显然功能太弱了。与其他编程语言类似,脚本语言应当也具有类似于“\emph{变量}”的功能实现“可拓展性”。

所谓“变量”,指的就是\emph{在运行过程中会发生变化的量},这些值可能是由用户输入给定的,或者在运行过程中生成的,或者是通过文件读取得到的。

\subsection{定义变量与初始化}\label{subsec:变量-定义变量与初始化}

与C语言等强类型语言不同,Shell脚本的变量在使用之前不需要对其进行“声明”,相对地则是需要对其进行\emph{初始化}。与其他编程语言类似,在Shell脚本中,第一次使用变量时需要对变量进行赋值(也可以叫做“初始化”)。例如,我们希望将字符串“Hello World!”赋值给一个变量,则可以使用\code{STRING="Hello World!"}

\begin{attention}
    与其他编程语言类似,在Shell脚本中,赋值也是使用\code{=}运算符。但不同的一点是,在运算符两侧\emph{不能有空格}。

    在变量命名时,需要遵守如下原则:变量名只能包括数字、字母和下划线(\code{\_}),第一个字符不能是数字,不能是已有的关键字\footnote{“关键字”指的是在Shell脚本中已经具有特定含义的词语,如\code{echo}就不能作为变量名。}。
\end{attention}

在一个程序中,可以同时存在多个变量,对于已经赋值的变量,也可以对其再次进行赋值(原有值会发生变化)。例如,下面的代码:

\begin{lstlisting}[language=bash,caption=variable,numbers=left]
#! /bin/bash
# 变量初始化
STRING1="Hello World!"
STRING2="I Like Bash"
STRING2="I Like Shell Script"
\end{lstlisting}

上述代码第3行定义了一个变量\code{STRING1},其赋值为\code{"Hello World!"};在第4行,首先定义了一个变量叫做\code{STRING2},首先赋值为\code{"I Like Bash"},在第5行又一次对其进行赋值,此时\code{STRING2}的值变为\code{"I Like Shell Script"}。

\subsection{调用变量}\label{subsec:变量-调用变量}

如果你熟悉其他编程语言如C、Java、Python等,也许在编写上面的语句时,你会很自然写出如\code{STRING2=STRING1}这样的语句。在你看来,这好像是把\code{STRING1}的值赋值给\code{STRING2},但当你运行时,发现事实并非如此。这是因为在Shell脚本中,变量的调用需要用到其他的方法。

\begin{attention}
    在上面我们提到\code{STRING2=STRING1}的含义是\emph{将\code{STRING1}的值赋值给\code{STRING2}},这对于了解过其他编程语言的读者而言是很自然的。但如果你没有学习过其他编程语言,需要特别注意的一点是:在程序设计语言中(几乎大多数的程序语言),\code{=}所表示的含义与你所熟悉的数学上的含义不同。在数学上,=表示一种\emph{状态},表达两个值相等;而在程序设计中,\code{=}表示将右边的值\emph{赋值给}左边的值(一种\emph{动作})。

    尽管在最后的结果上,二者是相同的,但数学上的=表达一种“状态”,而程序设计中表达一种“动作”,是不同的。一个很简单的例子就是上面的\code{STRING2=STRING1},从数学的角度看,这显然不成立,因为\code{"I Like Shell Script"}显然不可能\emph{等于}\code{"Hello World!"},但程序设计上是可行的,因为它表达了“赋值”的动作。

    也正因如此,在数学上,$a$和$b$相等写成$a=b$或者$b=a$都是可行的(这也就是等式的“对称性”);而对于程序设计而言,\code{a=b}和\code{b=a}显然是不同的,因为它们所表达的动作“方向”是不同的。
\end{attention}

在Shell脚本中,调用变量需要使用到\keyword{\$}符号。事实上,这不是你第一次见到它(如果你忘记了,请回到\ref{sec:简单for循环}一节,或者更准确的,\ref{subsec:简单for循环-关于变量}一节)。与定义变量不同,在Shell脚本中,但凡是需要\emph{调用变量}的地方,都需要使用\code{\$}符号。例如,在上面的例子中,如果你确实希望表达\code{STRING2=STRING1},需要写作\code{STRING2=\$STRING1}。

这里有一点“小绕”的地方在于,为什么在\code{STRING2}前面不需要添加\code{\$}符号。这是因为,我们实际上只是调用了\code{STRING1}变量的值,并不关心\code{STRING2}里面是\code{"I Like Shell Script"}还是\code{"I Like Roselia"}。因此,我们只需要通过\code{\$STRING1}来获得\code{STRING1}的值。

\begin{extend}
    你还可以做一个“不准确”的理解:\code{\$}总是视图将右边的内容“展开”为完整的内容。例如,假设\code{STRING2}的值为\code{"Hello World!"},那么在调用\code{STRING2=\$STRING1}时,可以写作\code{STRING2="Hello World!"}(将变量\code{STRING1}展开)

    对于前面所介绍的\code{for}循环,其本质是类似的。例如,\code{for i in \{1..5\}; do echo \$i; done},实际上也是将变量\code{i}展开为具体的1到5.

    当你了解这个时,对于后面的一些操作,会很有帮助。
\end{extend}

在了解了如何调用变量后,我们就可以做一些完整的事情了。例如,下面的一段完整代码实现了变量的初始化,修改赋值和调用,并在最后使用\code{echo}语句进行输出。

\begin{lstlisting}[language=bash,caption=variable,numbers=left]
#! /bin/bash
# 变量初始化
STRING1="Hello World!"
STRING2="I Like Bash"
# 修改变量的值
STRING2="I Like Shell Script"
# echo输出变量的值(调用变量)
echo $STRING1
echo $STRING2
\end{lstlisting}

运行上述代码,就可以看到输出了“Hello World!”和“I Like Shell Script”。

\subsection{字符串中的\$符号}\label{subsec:变量-字符串中的$符号}

像上面这样在字符串中使用\code{\$}符号,最简单的情况就是上面这种单纯输出一个变量。但大多数时候,我们可能希望在输出时提供更复杂的内容。例如,我们有下面的变量

\begin{lstlisting}[language=bash,caption=dollar\_in\_string]
#!/bin/bash
# 初始化变量
name="Jiaqi Z."
band="Roselia"
\end{lstlisting}

如果我希望输出“My name is Jiaqi Z., and I like Roselia”。如果考虑到\code{echo}可以\emph{连续输出多个参数},也许你会想写出\code{echo "My name is" \$name", and I like" \$band}这样的语句。确实,从运行的角度,这个句子没有任何问题。但显然从可读性的角度,稍显复杂。那么,有没有更简单,更清晰的方式呢?

在使用\code{\$}表示变量时,我们可以将变量名使用大括号将其括起来。例如,上面的例子,我们就可以写作\code{echo "My name is \$\{name\}, and I like \$\{band\}"}

\begin{extend}
    如果你尝试将大括号去掉,在这一例子中,同样可以运行出正确的结果。这是因为,\emph{每一个变量名后面都跟着一个“标点符号”}。如果对于再一般的情况,我们的变量名后面跟着另外一些字母。例如,如果我们在\code{\$name}后面再加个s,对于使用大括号的情况,则会输出\code{Jiaqi Z.s},而对于没有大括号的情况,则会输出错误的结果(由于没有变量叫做\code{names})。

    因此,为了更一般的情况,我们建议\emph{在使用变量名调用变量时加上大括号}。
\end{extend}

\subsection{变量简单运算}\label{subsec:变量-变量简单运算}

在前面的例子中,我们都是针对于字符串变量进行讨论。事实上,变量还可以保存一些数值信息(例如整数、小数等)。例如,我们可以定义变量\code{a=3}和\code{b=2.5}。

同时,在脚本中,我们也可以进行简单的四则运算。简单的方式则是利用\code{\$ (( 表达式 ))}的格式写出运算内容。例如,下面的代码则是简单计算1+1的结果:

\begin{lstlisting}[language=bash,caption=calculate,numbers=left]
#! /bin/bash
# 定义变量
a=1
b=1
# 计算
c=$(( ${a} + ${b} ))
# 输出
echo ${c}
\end{lstlisting}

在Shell脚本中,四则运算(加减乘除)的符号分别为\code{+}, \code{-}, \code{*}, \code{/},同时需要特别注意的两个符号是\code{\%}表示\emph{取余},即\emph{求得两个整数相除后的余数},例如,计算\code{echo \$(( 9\%4 ))}可以得到1;\code{**}(两个乘号)表示\emph{幂运算},例如,\code{echo \$(( 2**10 ))}表示$2^{10}$,即1024.

\begin{extend}
    在Shell脚本中,你可以使用这种方法进行整数的四则运算。对于小数而言,则需要使用更复杂的方法。例如,对于变量\code{a=1.5}和\code{b=2},如果希望做小数的四则运算,可以使用\keyword{bc}命令,写作\code{echo "\$\{a\} + \$\{b\}" | bc}。

    但对于脚本来说,通常你不应寄希望于它的运算功能(这主要由于运算效率的限制)。通常来说,对于需要使用复杂运算的任务,应当考虑其他效率更高的编程语言如C语言\footnote{在大多数Linux当中都有C语言的编译器\code{gcc},你可以使用\code{gcc --version}查看对应版本。}和Python语言。

    例如,上述问题如果希望使用C语言编写,则可以写作下面的代码:

    \begin{lstlisting}[language=C,caption=calculate.c,numbers=left]
#include "stdio.h"
int main()
{
    double a = 1.5;
    int b = 2;
    printf("%f\n",a+b);
    return 0;
}   
    \end{lstlisting}
    并通过\code{gcc calculate.c}的方式编译代码,得到\code{a.out}可执行文件。通过\code{./a.out}即可运行得到正确结果。

    正因如此,在本教程中,我们不会详细讨论脚本的计算(如果你确实需要使用脚本进行复杂运算,请查阅其他相关资料(例如\code{bc}和\code{awk}的相关使用方法)
\end{extend}

\subsection{错误处理}\label{subsec:变量-错误处理}
% 请在本节列出可能遇见的错误与解决方法

\subsubsection{-bash: <表达式> : syntax error: invalid arithmetic operator (error token is "<表达式>")
}

这可能是由于你错误使用了四则运算符,例如,在使用\code{\$(())}的方式进行计算时,要求只能进行整数四则运算。如果你的运算符两边出现了小数,则会出现错误。

\subsubsection{bash: <变量名>: command not found...}

这是因为你在对变量进行赋值时,在\code{=}两边加了空格。在Shell脚本中,赋值符号(\code{=})两边不能有空格,这一点与其他编程语言不同。