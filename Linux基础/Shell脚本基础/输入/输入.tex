% 请在下方的大括号相应位置填写正确的节标题和标签,以及作者姓名
\section{输入}\label{sec:输入}
\sectionAuthor{Jiaqi Z.}

% 请在下方的item内填写本节知识点
\begin{Abstract}
    \item 如何读取用户输入
    \item 如何读取命令参数
    \item 如何将命令执行结果赋值给变量
\end{Abstract}

在前面,我们仅仅讨论了脚本内定义的变量,这对于脚本而言远远不够。在实际使用脚本的时候,有一些信息只有在调用时才能知道。例如,如果我们希望编写一个可以删除文件的脚本\footnote{尽管我们已经有了\code{rm}命令,但我们可能还会有其他想法。例如,我们希望在删除完成后输出删除了哪些文件,或者我们希望将“删除”改为移动至某一个目录实现“回收站”的功能。},在编写时不可能知道需要删除哪些文件。因此,有必要在写脚本时考虑实现“交互”。

通常来说,交互的方式有三种:程序运行时输入、程序调用参数、以及外部文件。在本节,我们将讨论这三种交互方式如何在脚本中实现。

\begin{attention}
    严格来说,还有一种:管道输入。在本节我们不详细讨论管道的输入方式。
\end{attention}

\subsection{用户输入}\label{subsec:输入-用户输入}

在Shell脚本中,实现用户输入的方法是使用\keyword{read}语句。一般格式是\code{read <变量名>},例如,\code{read a}表示在运行时读取用户输入,并将输入结果赋值给变量\code{a}。

在调用\code{read}命令时,可以提供一个选项,\keywordin{read}{-p}表示在屏幕上显示提示信息,其格式为\code{read -p <提示信息> <变量名>}。

利用这一命令,我们可以实现简单的交互。例如,我们可以写一个简单的“应声虫”小程序,即当用户输入一个内容后,程序原封不动将其输出。

\begin{lstlisting}[language=bash,caption=my\_echo,numbers=left]
#!/bin/bash
# 读取输入
read -p "Please input a string: " STRING
# 输出
echo "You said: ${STRING}."
echo "Good Luck!"
\end{lstlisting}

其中,第3行我们使用\code{read}命令读取用户输入,并将其赋值给\code{STRING}。之后在第5行使用\code{echo}语句输出了用户的内容(在前面加了一些其他内容)。

在运行时,程序会首先输出\code{Please input a string: }并等待用户输入。当用户输入完毕后,按下回车表示完成,此时程序执行后面的内容(输出)。

\begin{attention}
    在读取输入时,不要在变量名前面“画蛇添足”加上一个\code{\$}符号。如果你试着这样做,会得到错误的结果——它有可能会输出空白信息,或者输出一些其他的内容。
\end{attention}

\begin{extend}
    输出空白或者其他信息取决于你的运行方式是使用\code{source}还是添加执行权限。对于前者,\code{source}本质上相当于在当前Shell终端下执行了脚本里的命令,其变量会延续到脚本外。因此,如果你在刚开始正确时输入了一个内容,如“Hello World!”,脚本会将其赋值给\code{STRING}变量并延续到Shell外部(用更专业的说法,这种“延续”实际上是“作用域”的体现)。此时如果你尝试在外面直接运行\code{echo \$STRING},也会得到对应的结果。

    如果你是添加了执行权限并运行的话,脚本实际上是在当前Shell下新建了一个“子Shell”并运行,运行过程中产生的变量仅会在这一Shell内有效(表现为程序内),当退出脚本时,变量也就因此失效了。

    在使用\code{read}输入变量时,如果后面加了\code{\$}符号,则不会输入任何内容。此时在\code{echo \$STRING}时,则会根据目前环境下已有的变量,输出对应的结果(已有的内容或空白)
\end{extend}

\subsection{参数输入}\label{subsec:输入-参数输入}

除了使用前面所介绍的\code{read}方法在程序运行时读取用户输入,在有些时候可能也希望通过类似于参数调用的方式输入。可以类比一下最开始我们所接触到的如\code{cd}和\code{rm},在切换目录或者删除文件时,都是在调用时直接给出对应的文件名,而不是在运行过程当中输入。那我们有没有类似的方法实现这一功能?

答案肯定是有的,而且这一部分你不需要任何特殊的命令。因为在脚本中,如果调用时提供了一个参数,默认在程序中就是\code{\$1},以此类推,如果有两个或多个参数,则分别是\code{\$2, \$3, \$4}等等。例如,下面的代码则实现了位置参数的调用:

\begin{lstlisting}[language=bash,caption=loc\_parameters,numbers=left]
#!/bin/bash
# 输出第一个参数
echo "First parameter is ${1}"
# 输出第二个参数
echo "Second parameter is ${2}"
\end{lstlisting}

当调用时,类似于之前使用其他内置命令那样,可以往其中传递参数(使用空格分割),如\code{./loc\_parameters hello world},则第一个参数为“hello”,第二个参数为“world”。

\begin{attention}
    通常来说,我们将这种形式上如\code{\$n}的参数叫做\emph{位置参数}。也请务必注意的一点是:位置参数默认是从1开始而不是从0开始的。当参数数量在9个及以内时,可以直接使用\code{\$1}到\code{\$9}这种形式,但如果到了10个及以上参数,需要在数字外面加大括号如\code{\${10}}。

    你也可以尝试在程序中使用\code{\$0},它表示\emph{正在运行的脚本名称}。
\end{attention}

\begin{extend}
    如果你真的尝试输出\code{\$0}的值,可能会意外地输出一个叫做\code{-bash}的内容而不是脚本名称。这是因为如果你使用的是\code{source}方式运行,在这种情况下,你的脚本实际上是在当前命令行环境下运行,此时程序中的\code{\$0}与你直接在命令行中输入\code{\$0}运行结果应当是一致的。

    另外,如果你的程序是在其他目录下运行(假设你已经将这一目录添加进环境变量),此时\code{\$0}会输出这一脚本所在的完整目录。
\end{extend}

除了直接使用位置编号表示参数本身外,Bash脚本还提供其他内置的参数表示其他信息。常见的例如\code{\$\#}表示传递的参数个数(不包括\code{\$0}),\code{\$\@}表示整个参数列表。下面的程序使用\code{for}循环遍历了所有参数(完整的\code{for}循环教程在后面介绍)

\begin{lstlisting}[language=bash,caption=special\_parameters,numbers=left]
#!/bin/bash
# 输出一些特殊参数
echo "Current file is ${0}"
echo "We have $# parameters"
echo "They are:"
for i in $@ 
do
    echo $i
done
\end{lstlisting}

其中,第6行使用\code{\$\@}符号表示传入的参数列表,并对其中的所有元素进行遍历(输出)

\subsection{读取命令作为输入}\label{subsec:输入-读取命令作为输入}

除了在运行时输入,在很多时候我们需要借助于一些命令读取文件的内容,并将其作为变量进行处理。例如,我们希望读取\code{INCAR}文件中的\code{ENCUT}所在的一行,根据前面所学习的方法,我们可以使用\code{grep}命令,如\code{grep ENCUT INCAR}命令来输出这一行。如果我们希望将这一命令作为变量输入到脚本中,只需要使用\code{\$(grep ENCUT INCAR)}这种形式即可,其命令使用小括号,且前面加上变量的\code{\$}符号。同样,在命令里面也可以使用变量以实现更复杂的交互功能,下面的代码实现了\emph{用户输入一个字符串和文件名,查找文件中包含这一字符串所在一行的内容}:

\begin{lstlisting}[language=bash,caption=print\_string,numbers=left]
#!/bin/bash
# 读取字符串
read -p "Please input a string: " STRING
# 读取文件名
read -p "Please input a file name: " FILE
# 读取命令并赋值
result=$(grep ${STRING} ${FILE})
echo ${result}
\end{lstlisting}

\begin{attention}
    上述代码并不是一个“完美”的代码,因为在读取文件时,并没有对文件是否存在进行检查。因此,如果输入了一个不存在的文件名,则会输出错误的结果(如同你正常使用\code{grep}时输入了错误的文件名那样报错)。当你测试这一段代码时,请提前创建好一个对应的文件。

    在后面的学习中,我们将进一步完善这一代码(设置一段代码实现文件是否存在的检查)
\end{attention}