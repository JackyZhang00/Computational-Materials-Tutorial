% 请在下方的大括号相应位置填写正确的节标题和标签,以及作者姓名
\section{第一个脚本}\label{sec:第一个脚本}
\sectionAuthor{Jiaqi Z.}

\begin{Abstract}
    \item 什么是Shell脚本
    \item 如何编写第一个Shell脚本——Hello World
    \item 如何运行脚本
\end{Abstract}

在前面的学习中,我们已经了解如何使用Linux命令在Shell进行操作,我们了解了如何对文件和目录进行简单的操作(如删除、复制等),同时我们也了解了一些更复杂的操作,例如使用\code{grep}和\code{sed}进行文本的查找和替换等。最后,我们还了解了如何使用\code{for}循环来进行批量操作。

而在一些特殊的场景下,我们可能希望做更复杂的操作,或者说,我们希望更简单地执行一些操作(这两句话本质上是一样的)。例如,如果我们每次都需要使用\code{sed}命令修改特定的内容,如\code{ENCUT = 400},刚开始还好,时间长了可能就会“嫌麻烦”。此时可能就会希望有一个单独的命令(比如叫做\code{changeENCUT})来实现这一功能。而实现这一功能的方法,就是使用\emph{脚本}。

在本章,我们将会讨论如何编写自己的脚本。类似于编程语言,脚本里面将会包含大量的编程思想——如输入、输出、条件、判断、函数等。在学习这一部分之前,希望你已经有了部分编程语言基础(没有也没有关系)。

\begin{extend}
    关于Shell和Bash的区别:一般来说,Shell指的是系统和用户交互的那层“外壳”,之前我们所学习的内容,其操作都是在Shell当中进行的。Shell具有多种版本,如“Bourne Shell”、“Bourne Again Shell”、“C Shell”等。其中,“Bourne Again Shell”就是我们所谓的“bash”。在你的操作系统下,可以使用\code{top}命令查看其Shell类型。
    
    Shell脚本,全称叫做“Shell Script”,是一种在Shell当中批量运行多条语句的程序。
    
    由于目前主流的Shell是基于Bash解释的,而我们所写的Shell脚本,实际上也大多都是Bash脚本,因此在后文当中,我们可能不会精确区分Bash脚本和Shell脚本的区别。
\end{extend}

\subsection{编写第一个脚本}\label{subsec:第一个脚本-编写第一个脚本}

正如任何程序的开始都是“Hello World”,在本章我们也不例外。在Linux当中编写Shell脚本不需要额外的程序,只需要使用\code{vi}编写一段文本文件,并\emph{赋予它运行权限},就可以作为脚本运行了。首先通过\code{vi}创建一个名为\code{hello}的文件,并输入如下内容:

\begin{lstlisting}[language=bash,caption=hello,numbers=left]
#!/bin/bash
# 输出Hello World!
echo "Hello World!"
\end{lstlisting}

编写完成后保存,并添加运行权限(\code{chmod +x hello},详见第\ref{sec:文件权限管理}一节),然后执行\code{./hello},即可在屏幕上看到输出结果。

\begin{attention}
    在运行时需要加上\code{./}表示在当前目录寻找命令。在Linux当中,不添加\code{./}表示在环境变量\code{PATH}下查找文件运行,你可以在家目录下找到\code{.bashrc}的文件,里面包含有一系列配置Bash的命令,其中就有对环境变量的设置。

    在运行前,你需要保证程序已经具有运行权限,或者可以使用\code{source ./hello}或\code{. ./hello}的方式(二者等价)“临时赋予运行权限”\footnote{这是表面上的用法,事实上,\keyword{source}本意是\emph{在当前Shell下运行文件}。相对的,其他的用法(不加\code{source})则是在当前shell下新建了一个“子Shell”运行代码,其运行结果(例如一些变量)并不会带回外面。}。
\end{attention}

其中,代码第一行\code{\#!/bin/bash}表示\emph{使用bash运行}。正如前面所说的那样,Shell具有多种版本,因此,在编写时应当特别指定你所使用的版本。由于目前大多数Shell都是使用Bash,因此这一行在有些时候“可以省略”。但我们不建议将其省略,因为\emph{你永远不能保证你的这个脚本今后会在哪个版本的Shell下运行}。

\begin{extend}
    你可以想见,\code{/bin/bash}就是\code{bash}命令所在位置,你可以去看一下是不是真的存在。在查看的时候,注意是从“根目录”开始而不是“家目录”

    如果你真的这么做了,一种简单的方法是在\code{/bin/}目录下使用\code{ls | grep bash}只输出具有“bash”的文件,从而简化输出结果。当然,你也可以直接使用\code{ls bash}查看。

    当然,不建议你尝试使用\code{vi bash}查看里面的内容,\emph{它不是文本文件}。
\end{extend}

代码第二行以\code{\#}开头表示\emph{注释}。如同编写其他代码一样,使用注释是一个好习惯,它可以帮助你划分代码段落,以及记住对应的功能。随着学习的深入,我们会编写越来越长的脚本。因此,记得加注释是个好习惯。

第三行是这一脚本的关键,它使用\keyword{echo}实现字符串的输出。事实上,Shell脚本的每一个命令都可以在Shell本身下运行。因此,你也可以直接在Shell运行这一命令,会实现同样的效果。而编写脚本之后,就可以直接通过\code{./hello}实现这一功能,这便是脚本的作用。

\begin{attention}
    \code{echo}会将它后面的所有内容输出(在其他一些编程教材中,会将这一功能叫做“应声虫”,实际上,echo也就是“回音”的意思)。我们在这里添加双引号是为了强调它们是整体的,事实上,当你去掉这两个双引号,对程序运行结果没有任何影响。

    如果你在你的脚本中编写了代码,并同样使用了双引号,请注意:\emph{使用英文符号而不是中文符号},这一点在后续所有脚本编写过程中都应当注意。一般来说,我们不建议在脚本当中添加中文,虽然你写\code{echo "你好,世界!"}可能也会得到正确的结果,但不会永远如此。

    在本章的教程中,为了考虑到读者水平,我们的注释部分都会采用中文,如果这样也会引起脚本运行的失败(在测试时正常,但不敢保证在你的电脑也会正常),请删除中文注释后运行。
\end{attention}

\subsection{添加至环境变量}\label{subsec:第一个脚本-添加至环境变量}

正如前面所说,使用\code{./hello}表示在当前目录下查找名为\code{hello}的脚本并运行。这可以帮助我们快速调试代码,但在真正应用时,我们可能会希望在任何目录下运行脚本。此时就会希望将代码添加至环境变量,也就是前面所说的\code{PATH}。添加方法有两种——将脚本放置到已有的环境变量中,或者将脚本所在的目录设置为环境变量。

你可以通过\code{\$PATH}命令输出当前环境变量,通常来说,你可以将你所编写的脚本命令放置在\code{\texttilde/bin/}目录下(这一般都是用户的环境变量)完成后,你可以在任何目录下运行你的脚本了(不需要\code{./}了)。例如,在完成上述配置后,在任何目录下运行只需要使用\code{hello}即可。

如果你编写了一系列脚本,一个简单的方法是直接将它们所在的目录设置成“环境变量”。此时需要通过\code{.bashrc}文件。假设你的脚本所在目录为\code{\texttilde/bash/},使用\code{vi}打开\code{.bashrc}文件,在最后一行添加\code{export PATH=\$HOME/bash:\$PATH}即可。

其中,\keyword{export}是用来设置“环境变量”的命令,后面的\code{PATH=...}则是“变量赋值”的过程(后面就会学到)。\code{\$HOME}是系统内置的变量,表示\emph{用户的家目录},你可以在Shell下使用\code{echo \$HOME}查看变量的值\footnote{在这里我们又不知不觉接触到\code{echo}的新用法:输出变量的值。这本是后面的内容,在这里你可以提前先了解一下。},后面的\code{\$PATH}则表示原先的环境变量。

简单说,这一语句的意思就是在原有的\code{PATH}变量前面添加一个新的\code{\$HOME/bash}。添加完成后你需要使用\code{source \texttilde/.bashrc}命令“激活”这一环境变量(或者重启也可以实现这一功能),然后即可在任何地方如一般运行命令一样运行你在\code{\texttilde/bash/}目录下所有的脚本了。

\begin{attention}
    在后面的教学演示中,我们都不会添加\code{./}运行脚本(或者说,只有在这一节我们会详细提到如何运行脚本,后面都简单说作“运行脚本”)。如无特殊说明,无论是哪种方法(在当前目录、添加环境变量),运行最终效果都是一样的,后面不再赘述。

    为了你的方便,建议新建一个目录作为你后续练习脚本的目录,并使用上面的方法将其添加到环境变量中。
\end{attention}


\subsection{错误处理}\label{subsec:第一个脚本-错误处理}

\subsubsection{-bash: <脚本名>: Permission denied}

这可能是因为你在运行脚本时没有赋予其运行权限而直接运行脚本,如果你没有赋予权限,请使用\code{source}命令或\code{.}运行脚本。这同样适用于环境变量中的命令。对于环境变量中没有运行权限的脚本,需要使用\code{source <脚本名>}或\code{. <脚本名>}执行。