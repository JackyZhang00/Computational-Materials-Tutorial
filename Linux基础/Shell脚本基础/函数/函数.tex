% 请在下方的大括号相应位置填写正确的节标题和标签,以及作者姓名
\section{*函数}\label{sec:函数}
\sectionAuthor{Jiaqi Z.}

% 请在下方的item内填写本节知识点
\begin{Abstract}
    \item 如何定义与调用函数
    \item 函数参数
    \item 函数返回值
\end{Abstract}

前面我们已经学习了程序的基本结构——顺序语句、条件语句和循环语句。此时,应当已经足够完成大多数脚本程序的编写。在本节和下一节,我们将介绍一些扩展性的内容——函数和数组。这两部分可以帮助你更快、更方便地编写程序。

\subsection{定义函数}\label{subsec:函数-定义函数}

在bash脚本中,函数的定义必须在调用之前。这是因为脚本语言是按照顺序执行的,如果在程序不知道的情况下直接调用,那必然会报错。在bash脚本中,对一个函数定义的基本格式为:

\begin{lstlisting}[language=bash]
[function] function_name()
{
    函数体;
    [return int.;]
}
\end{lstlisting}

在定义时,可以在前面加上\code{function}以示区分,也可以不加直接以函数名作为开始。在函数内部的语句块使用大括号括起来(函数体),在函数的结束可以使用\code{return}语句返回一个范围在0-255之间的整数。

\begin{attention}
    与大多数程序语言不一样,bash脚本对函数返回类型有明确的数值要求,且这个要求是不随程序员所改变的。如果要返回函数运行结果是否成功,可以使用\code{return}语句,但如果返回的是其他内容(例如返回两个数字相乘的结果),则可以使用字符串的形式返回。

    在bash的函数定义中,我们不会在函数开始的地方强调参数的个数和类型(这点与C语言等类似语言不同),具体的参数将在函数体内部使用\code{\${num}}的形式体现(与\ref{subsec:输入-参数输入}所介绍的参数类似)
\end{attention}

例如,下面的程序简单定义了一个名字叫做\code{my\_echo}的函数:

\begin{lstlisting}[language=bash,numbers=left,caption={my\_echo\_function}]
#!/bin/bash
# 定义函数
my_echo()
{
    echo "Hello"
}
\end{lstlisting}

这个函数内部只有一个输出语句,且没有返回值。

\begin{extend}
    我们这里所说的“没有返回值”,特别指的是不使用\code{return}返回的返回值。你也完全可以将这个函数的\code{echo}输出看作是一个字符串返回。
\end{extend}

\subsubsection{有参数函数的调用}

与\ref{subsec:输入-参数输入}一节所介绍的方法类似,在函数内部我们也可以使用\code{\$1}表示第一个参数,当参数个数大于等于10时,则需要使用大括号将数字括起来,例如,\code{\$\{10\}}表示第10个参数。

在函数体内部,我们也可以使用\code{\$\#}表示参数的个数(与前面所介绍的方法一样),可以使用\code{\$\*}将所有参数以字符串的方式输出。例如,下面的程序就实现了两个整数相加的计算函数:

\begin{lstlisting}[language=bash,numbers=left,caption={add\_function}]
#!/bin/bash
# 两个整数相加
function add_function() {
    c=$(( $1 + $2 ))
    echo $c
}
\end{lstlisting}

这里的第4行我们使用\code{\$1}和\code{\$2}分别表示第1个参数和第2个参数,并将计算结果赋值给变量\code{c},并在最后输出变量\code{c}的值。

在这里请思考一下:为什么我们的程序使用\code{echo}语句输出返回值,而不是使用\code{return}语句返回结果?

\answer{因为两个数字相加的结果可能超过255。}


\subsection{函数调用}\label{subsec:函数-函数调用}

在定义函数的基础上,调用函数就可以将函数看作是一个新的\emph{命令}使用了。例如,对于我们定义的\code{my\_echo}函数,在程序内就可以直接使用\code{my\_echo}命令调用这个函数,从而输出“Hello”。

对于下面的\code{add\_function}函数,类似于前面在介绍命令行传递参数一样,我们也可以使用类似的方式调用这个函数。下面的代码则完整展示了函数定义与调用的全过程:

\begin{lstlisting}[language=bash,numbers=left,caption={add\_function}]
#!/bin/bash
# 两个整数相加
function add_function() {
    c=$(( $1 + $2 ))
    echo $c
}
result=$(add_function 200 300)
echo ${result}
\end{lstlisting}

我们在第7行使用了一个赋值语句特别说明了\emph{如何将函数\code{echo}的返回值作为一般的返回值处理},也可以看到,对于这个例子,当计算200+300时,程序可以给出正确的结果(500)。

在一般使用时,我们也可以直接省略掉后面两行,而直接使用\code{add\_function 200 300}也能输出类似的结果,当然,本例所给的方式更具有可扩展性。例如,你可以对这个结果进行进一步的计算,例如判断它是否大于255.(请试着完成这段代码)

最后,让我们再稍微扩展一下,如果用户在最开始给定一个数字个数,使用上面的函数,实现多个整数相加的结果。例如,用户一开始输入5表示有5个整数进行相加,然后会依次输入5个整数,程序输出5个数字相加的结果。我们在这里给出最终答案,请试着理解这段代码,并在理解后尝试自己写出来实现相同的效果:

\begin{lstlisting}[language=bash,numbers=left,caption={add\_function2}]
#!/bin/bash
# 多个整数相加
function add_function() {
    c=$(( $1 + $2 ))
    echo $c
}

read -p "Please input a number" count
index=1
a=0
while [ $index -le $count ]
do
    read b
    result=$(add_function $a $b)
    a=$result
    ((index++))
done
echo $result
\end{lstlisting}

在分析这段代码时,请务必考虑的内容是:\code{while}内部的循环条件为什么是这样(如果写成\code{-lt}会发生什么);为什么第15行要有一句\code{a=\$result},它的作用是什么?在整个程序中,\code{index}变量的作用是什么,为什么它的初始值设置为1;同时,为什么在开始时\code{a}有一个初始为0的值?

当你想明白这些答案时,相信这段代码就没有难度了。当然,我们这里是逐一计算的,那有没有办法在存下多个数据后一次性计算呢?这里的难点在于如何存储多个数据,这就需要下一节\emph{数组}相关的内容了。

% \subsection{错误处理}\label{subsec:节标题-错误处理}
% % 请在本节列出可能遇见的错误与解决方法

% \subsubsection{错误1}

% \subsubsection{错误2}

% \subsubsection{错误3}