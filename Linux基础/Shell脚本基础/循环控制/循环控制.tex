% 请在下方的大括号相应位置填写正确的节标题和标签,以及作者姓名
\section{循环控制}\label{sec:循环控制}
\sectionAuthor{Jiaqi Z.}

% 请在下方的item内填写本节知识点
\begin{Abstract}
    \item 如何使用\code{break}语句
    \item 如何使用\code{continue}语句
\end{Abstract}

在\ref{sec:循环}一节中,我们曾简单提到了“死循环”问题。在当时看来,死循环是一个应当避免的“错误”,但事实可能不总是如此——在有些情况下,我们可能不得不希望使用死循环。例如,我们希望一直读取用户的输入,直到用户输入0时退出。此时当然可以使用\code{while}循环或者\code{until}循环来解决这一问题。但还有一种思路——设计一个死循环,当用户输入0时跳出循环。

\begin{extend}
    在其他领域,死循环可能是更常见的。例如,在一些单片机系统中,会使用死循环来执行对应的任务,例如读取传感器数据、显示处理数据等;而在Windows操作系统中,任何一个窗口在创建时都会启动叫做“消息循环”的死循环来保持窗口——对于Windows来说,如果没有这个死循环,窗口会立即关闭。

    因此,在程序中,死循环并不是“绝对的错误”,而要根据需要选择。一些可控的、必需的死循环也是程序所必要的一部分。
\end{extend}

回到最开始的例子,在bash语言中,我们有两种方式对循环语句进行控制——\code{break}语句和\code{continue}语句。其中前者表示“终止循环”,而后者表示“继续循环”



\subsection{使用\code{break}停止循环}\label{subsec:循环控制-使用break停止循环}

\begin{extend}
    本节所要介绍的\code{break}与\code{continue},在使用方式与效果上与C语言的\code{break}和\code{continue}\emph{完全相同}。因此,如果你足够熟悉C语言,本节完全可以跳过。但我们还是建议你通过阅读这一节复习一下相关的内容。
\end{extend}

\code{break}语句的作用是用来\emph{跳出循环},例如,下面的程序,用户输入一系列数字,当用户输入0时,循环结束,输出“end”

\begin{lstlisting}[language=bash,numbers=left,caption={break\_example}]
#!/bin/bash
# break跳出循环
read a
while [ true ]
do
    if [ $a -eq 0 ]; then
        break
    fi
    echo $a
    read a
done
echo "end"
\end{lstlisting}

在第4行,我们设置了一个条件\code{true},使循环条件始终成立,从而令其成为一个“死循环”。在第6行利用\code{if}语句判断输入的变量是否为0,如果是0,则通过\code{break}跳出循环,执行最后一行的“end”输出。如果输入的数字不为0,则输出对应的变量值。

\begin{attention}
    请务必仔细思考一下,程序的第10行又写了一句\code{read a}有什么作用?如果没有这句命令会发生什么?在程序中有两句\code{read}命令稍显繁琐,有没有简单的方法使用一句即可实现相同的功能?
\end{attention}

除了在\code{while}循环外,\code{for}循环和\code{until}循环也可以使用\code{break}语句。例如,下面的程序遍历了10-20之间所有的数字,且当数字为7的倍数时停止。

\begin{lstlisting}[language=bash,numbers=left,caption={break\_example2}]
#!/bin/bash
# 在for循环中使用break语句
# 当遍历到7的倍数时停止
for i in {10..20}
do
    if [ $(( $i%7 )) -eq 0 ]; then
        break
    fi
    echo $i
done
echo "end" 
\end{lstlisting}

其中,变量\code{i}会依次从10递增到20,当\code{i}取值为14时,满足7的倍数\footnote{这里使用常见的方法——“取余”判断倍数,具体的计算方法可以查看\ref{subsec:变量-变量简单运算}一节。},从而进入条件判断内,执行\code{break}命令,跳出循环输出\code{end}

\begin{attention}
    上面的两个例子还说明了,\code{break}跳出循环是“立刻跳出”,即如果循环体内有剩下的其他语句也不会执行了。例如,上面的判断倍数的例子,当\code{i}取值为14时,也没有输出这个值(因为输出命令在\code{break}后面。

    可以尝试一下:如果将\code{echo \$i}放在循环体内第一行,输出结果会有什么不同?
\end{attention}

\subsection{使用\code{continue}跳过循环}\label{subsec:循环控制-使用continue跳过循环}

与C语言类似,在bash脚本中,也可以使用\code{continue}跳过循环。这里所说的“跳过”,指的是\emph{跳过当前轮次的循环}。让我们直接来看一个例子——假设将上面判断7的倍数的例子中的\code{break}改成\code{continue},看看会发生什么。

\begin{lstlisting}[language=bash,numbers=left,caption={continue\_example2}]
#!/bin/bash
# 使用continue语句跳过循环
for i in {10..20}
do
    if [ $(( $i%7 )) -eq 0 ]; then
        continue
    fi
    echo $i
done
echo "end" 
\end{lstlisting}

运行后可以看见,程序输出了10-20的所有数字,除了14。这是因为当变量\code{i}为14时,条件满足,执行\code{continue}语句,跳过了当前轮次的循环,直接进入下一个循环轮。与\code{break}类似,这里也不再执行后面的语句(即不会输出14这个数字)

虽然\code{continue}在语法上和\code{break}类似,难度也不大,但其使用时存在一个非常潜在的“隐患”。让我们再来看一下本节最开始的程序,如果将其中的\code{break}改成\code{continue},会发生什么?此时代码长成下面这样:

\begin{lstlisting}[language=bash,numbers=left,caption={continue\_example}]
#!/bin/bash
# 使用continue跳过0的判断,继续输入?
read a
while [ true ]
do
    if [ $a -eq 0 ]; then
        continue
    fi
    echo $a
    read a
done
echo "end"
\end{lstlisting}

一些“粗心大意”的人可能会认为:这段代码一直读取用户的输入并输出对应的内容,当用户输入0时跳过输出。让我们执行一下这段代码,当用户输入非0的数字时,程序很正常地输出对应的数字;当用户输入0时,程序也确实跳过了0的输出;但是,\emph{从这开始程序再也读取不了输入了}。从程序代码的执行过程可以很容易理解这一点,当用户输入0时,此时变量\code{a}的值为0,进入条件判断内,跳过当前循环。正如前面所说的那样,程序跳过循环\emph{不再执行后面的语句}。因此,在后面的循环中变量\code{a}始终保持0的值,即始终在4-7行语句间循环。

\begin{attention}
    对于\code{for}循环而言,由于它是对序列进行遍历,一般情况下总是“有限”的,因此循环总是能停止的。而对于\code{while}和\code{until}循环而言,在循环开始和\code{continue}中间,\emph{要有改变循环条件或者终止循环的语句},否则循环将陷入“死循环”。
\end{attention}


% \subsection{错误处理}\label{subsec:节标题-错误处理}
% % 请在本节列出可能遇见的错误与解决方法

% \subsubsection{错误1}

% \subsubsection{错误2}

% \subsubsection{错误3}