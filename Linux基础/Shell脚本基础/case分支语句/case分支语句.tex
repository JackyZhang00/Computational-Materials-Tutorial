% 请在下方的大括号相应位置填写正确的节标题和标签,以及作者姓名
\section{\keyword{case}分支语句}\label{sec:case分支语句}
\sectionAuthor{Jiaqi Z.}

% 请在下方的item内填写本节知识点
\begin{Abstract}
    \item 如何使用\code{case}语句实现分支判断
\end{Abstract}

在前面的\code{if}语句中,已经了解了如何进行判断。理论上了解了\code{if}语句后可以解决所有分支情况,但是有一些情况可能比较特殊。例如,我们希望实现一个菜单界面,此时可能需要用户输入一些指令表示对应的功能。可以想见,如果使用\code{if}语句,将会有许多的判断条件,甚至随着条件的增多,也会影响程序运行效率\footnote{这是因为在进行判断时,脚本需要对所有条件进行遍历。}。

因此,我们希望可以寻找一种方法,直接对变量进行判断,并根据它的值选择合适的语句执行。可以猜到,这就是本节\code{case}语句所要解决的问题。

\subsection{\code{case}语句基本结构}\label{subsec:case分支语句-case语句基本结构}

类似于\code{if}判断语句,\code{case}判断语句的基本结构为:

\begin{lstlisting}[language=bash]
case <变量> in
pattern1)
    statements1
    ;;
pattern2)
    statements2
    ;;
......
*)
    statements3
    ;;
esac
\end{lstlisting}

程序会根据变量所在(in)的模式(可以是一个单独的值,或者正则表达式),选择合适的语句进行执行。例如,如果变量满足\code{pattern1},则会执行\code{statements1}语句;如果满足\code{pattern2},则会执行\code{statements2}语句;以此类推,如果所有的都不满足,则执行\code{*}所对应的\code{statements3}语句。

\begin{attention}
    有几个细节需要特别关注:所有匹配模式都是以小括号作为结束分割(没有左括号);当每一个分支里面的语句结束时,需要有\code{;;}作为结束的标志;在\code{case}语句结束时,需要有\code{esac}作为结束标志。
\end{attention}

\begin{extend}
    正如在\ref{sec:判断语句}中所讲的那样,\code{esac}也是\code{case}的倒序。另外,模式当中的\code{*}实际上可以想见,表示的就是通配符里面的可以表示任意多个字符的\code{*}符号。
\end{extend}

\subsection{\code{case}语句应用}\label{subsec:case分支语句-case分支语句应用}

下面让我们来看几个常见的应用例子。

\subsubsection{单一字符匹配}

在\code{case}语句中,最简单的就是对单独一个字符进行匹配。下面这个例子则实现了一个对菜单的模拟,用户通过输入1或者2来实现对应不同的功能。

\begin{lstlisting}[language=bash,caption=simple\_menu,numbers=left]
#!/bin/bash
# 模拟菜单选项
echo "--------------------"
echo "1) option 1"
echo "2) option 2"
echo "--------------------"
read -p "Please input a number: " number
case $number in
1)
    echo "You input 1"
    echo "I will do something for option 1..."
    ;;
2)
    echo "You input 2"
    echo "I will do something for option 2..."
    ;;
*)
    echo "You did NOT input 1 or 2"
    ;;
esac
echo "Bye!"
\end{lstlisting}

运行上面的代码,可以看见:当用户输入1时,程序可以直接跳转到\code{1)}所对应的语句;对应的,当用户输入2时,可以跳转到\code{2)}所对应的语句;如果输入其他内容(例如输入3),则会跳转到最后提示输入错误。

\begin{attention}
    与\code{if}语句类似,\code{case}语句的前后,即开始的提示和后面的“Bye!”无论哪种情况都会输出,因为它们不属于\code{case}语句内。
\end{attention}

\subsubsection{字符串匹配}

字符串匹配与单一字符匹配完全一样(你可以将单个字符理解成一种特殊的字符串)。但是,在这里我们也有一点新东西要讲。如果我们希望多个选项匹配同一个分支,可以\emph{使用\code{|}符号表示“或”进行分隔}。例如,下面的程序则是根据用户输入的字符串选择特定的分支(有时多个输入对应同一个分支):

\begin{lstlisting}[language=bash,caption=favourite\_band,numbers=left]
#!/bin/bash
# 字符串匹配
read -p "Please input you favourite band: " band
case $band in
"PPP"|"ppp"|"Poppin'Party")
    echo "Thanks! I have known that you like Poppin'Party!"
    ;;
"Roselia"|"R")
    echo "Thanks! I have known that you like Roselia!"
    ;;
"RAS"|"RAISE A SUILEN"|"Raise a suilen")
    echo "Thanks! I have known that you like RAISE A SUILEN!"
    ;;
"MyGO!!!!!"|"mygo"|"mygo!!!!!")
    echo "Thanks! I have known that you like MyGO!!!!!"
    ;;
*)
    echo "Sorry...I don't know this band."
    echo "But now I have known it -- $band"
    echo "Thank you!"
    ;;
esac
echo "Bye!"
\end{lstlisting}

与前面的代码分析完全类似,但特别的是,这里面分支的判断使用\code{|}表示“或”,例如,“PPP”、“ppp”和“Poppin'Party”对应的都是同一个分支;类似地,“R”和“Roselia”也是对应同一个分支,当用户输入“R”或者输入“Roselia”时,得到的结果是一样的。

\subsubsection{正则表达式匹配}

正如一开始所说的那样,在匹配时可以使用\emph{正则表达式}。例如,下面的代码试图读取用户输入的参数为字母还是数字\footnote{为了复习之前的变量种类,我们在这里尝试使用参数给定变量而不是使用\code{read}命令。}:

\begin{lstlisting}[language=bash,caption=number\_or\_letter,numbers=left]
#!/bin/bash
# 读取调用时的选项
export LC_ALL=C
case $1 in
[a-z])
    echo "You input a lowercase"
    ;;
[A-Z])
    echo "You input an uppercase"
    ;;
[1-9])
    echo "You input a number"
    ;;
*)
    echo "You input other things"
    ;;
esac
echo "Bye!"
\end{lstlisting}

\begin{extend}
    也许你足够灵敏,注意到了第3行的\code{export LC\_ALL=C},这句命令表示将程序运行语言环境设置为默认值(可以通过\code{locale}查看环境语言设置),其中\code{C}表示系统默认值。
    
    在这里我们需要添加这一行命令以确保程序运行正确,但这行命令并不影响我们对\code{case}的理解。

    同时,当你运行这一行代码之后,可能程序中的中文注释发生乱码,这并不影响程序运行结果,你可以重新启动系统以恢复开始的状态。
\end{extend}

在调用时,当用户传递给一个小写字母时,则会匹配到\code{[a-z]},相对地,若给定一个大写字母作为参数,则会匹配到\code{[A-Z]}。但是,我们这里只能对用户输入的一个参数进行判断,如果用户输入多个字符的参数(例如\code{abc}),程序则会认为输入了其他内容。如何修改程序呢?请自己思考并尝试写一段代码\footnote{请真的这样做,这会极大提高你的编程思维。},同时自己测试它。(为简单起见,我们只需要匹配第一个字符即可)

同时,正则表达式也可以使用\code{|}作为分隔以进行多个情况的判断,例如,我们希望将上面的代码修改为无论大小写字母都判断为字母,则可以写成下面的样子:

\begin{lstlisting}[language=bash,caption=number\_or\_letter2,numbers=left]
#!/bin/bash
# 读取调用时的选项
export LC_ALL=C
case $1 in
[a-z]|[A-Z])
    echo "You input a letter"
    ;;
[1-9])
    echo "You input a number"
    ;;
*)
    echo "You input other things"
    ;;
esac
echo "Bye!"
\end{lstlisting}


\subsection{错误处理}\label{subsec:分支语句-错误处理}
% 请在本节列出可能遇见的错误与解决方法

\subsubsection{-bash: <文件名>: line <行号>: syntax error near unexpected token `)'}

这可能是因为你在结束分支时少了\code{;;}表示当前分支内结束。如果你了解C语言,你可能会很自然将这这个符号作为C语言中\code{switch-case}的\code{break}语句。但显然,C语言的\code{break}更加灵活(它可以没有从而越过其他分支),但Shell脚本不能没有\code{;;}结束。

\subsubsection{-bash: <文件名>: line <行号>: syntax error near unexpected token `<字符串>'}

这可能是由于你在使用\code{case}语句后忘记结束\code{esac}而后面还有其他语句从而报错。如果后面没有语句,则可能会出现下面的错误:

\subsubsection{-bash: <文件名>: line <行号>: syntax error: unexpected end of file}

这也表明你可能没有使用\code{esac}作为\code{case}语句的结束。