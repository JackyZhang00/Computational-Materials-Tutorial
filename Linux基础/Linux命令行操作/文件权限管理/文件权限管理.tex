% 请在下方的大括号相应位置填写正确的节标题和标签,以及作者姓名
\section{文件权限管理}\label{sec:文件权限管理}
\sectionAuthor{Jiaqi Z.}

% 请在下方的item内填写本节知识点
\begin{Abstract}
    \item 用户、用户组和其他用户
    \item 如何查看文件权限
    \item 如何修改文件权限
\end{Abstract}

% 请在正文相应位置填写正确的小节标题(或小小节标题),同时将标签的“节标题”和“小节标题”改为实际内容

\subsection{用户和用户组}\label{subsec:文件权限管理-用户和用户组}

Linux是一个多用户操作系统,因此,如何管理不同用户就成为一个至关重要的话题。例如,在科研过程中,同一课题组的多个成员可能会使用同一个服务器,此时每一个成员就是Linux当中的\emph{用户}。每一个用户通常都有一个主目录,通常为\code{/home/}下的目录\footnote{虽尽管我们经常说\code{/home/<用户名>}是用户的家目录,但实际上用户在创建时可以通过\keyword{useradd}命令的\keywordin{useradd}{useradd -d}选项指定主目录。但除非你是服务器管理人员(具有root账号),否则用不到该功能。}。

在\ref{subsec:目录操作-显示目录文件}当中介绍过如何使用\code{ls -l}查看一个文件的完整信息,其中提到了\emph{用户组}的概念。顾名思义,用户组就是\emph{用户的组合}。举一个例子:如果我们把你家庭的房子看作一个Linux操作系统的话,那么你的家人和你就组成一个用户组。而每一个人就是一个用户。对于家庭的共有物品而言(例如空调、冰箱等)是所有家人可以共同使用的,即对整个用户组可用;而相对地,你的房间,你的柜子可能只是你自己可以打开,此时我们说只对某一特定用户可用。

相对地,对于不是你家庭成员的其他人(比如你的邻居等),他们是属于其他用户组的用户,对于你家的所有东西都不可用。

\begin{attention}
    上面的例子也许你还看得“一头雾水”,什么可用、不可用,到底有什么用。事实上,用户组的应用场景大多集中在关于权限的操作上。而这件事则是下一部分的内容。

    同时,前面提到的所有用户中有一个特殊用户--\emph{root}用户。对于他而言,拥有最高的权限和能力,即可以进入任何地方。也正如在前面多次提到的那样,对于一般科研工作而言,不需要了解root的相关内容。因此在这里,我们就将其略过去了。
\end{attention}

\subsection{文件权限}\label{subsec:文件权限管理-文件权限}

前面介绍\keywordin{ls}{ls -l}命令时已经说明了一些关于文件权限的内容,现在来进一步介绍\emph{如何查看文件权限},以及\emph{如何理解文件权限的含义}。

与前面所介绍的一样,文件权限可以利用\code{ls -l}或者\code{ll}查看。通常来说,每一个文件(包括目录)的第一组字符串表示了文件类型和参考文献。其中文件类型的相关内容已经在\ref{subsec:查看文件-Linux文件类型}当中介绍过了,现在我们重点关注后面九个字符,即\emph{文件权限}。

以\code{/bin}目录下的\code{cd}文件为例\footnote{正如你所见的那样,\code{/bin/}目录下存放着所有系统命令。因此,Linux的命令行实际上就是调用了这些程序。},使用命令\code{ls -l /bin/cd}可以得到如下结果

\begin{lstlisting}[language=bash]
$ ls -l cd
-rwxr-xr-x. 1 root root 26 Oct  9  2021 cd
\end{lstlisting}

输出结果的第一个部分就是文件权限。其中第一个字符\code{-}表示这个文件是一个\emph{普通文件},后面的9个字符,每三个一组,分别表示\emph{拥有者},\emph{所属用户组}和\emph{其他用户}的权限。例如,对于\code{cd}文件为例,拥有者(即输出结果的第三部分\code{root})的权限是\code{rwx};而所属用户组(输出结果的第四部分\code{root})具有\code{r}和\code{x}的权限;同样,对于其他用户来说,也是具有\code{r}和\code{x}权限。

\begin{extend}
    在这里你见到了\keywordin{ls}{ls -l}的新用法,即在后面添加一个文件的路径。即\code{ls -l <文件路径>},可以查看该文件的属性。

    同时,也许你注意到上面的第一部分输出结果还有一个\code{.},这表明该文件是在SELinux模式下创建的。其中SELinux叫做安全增强型Linux(Security-Enhanced Linux),其目的在于最大限度地减小系统中服务进程可访问的资源\footnote{详细的内容可以参考网址https://blog.csdn.net/yanjun821126/article/details/80828908和https://docs.redhat.com/zh\_hans/documentation/red\_hat\_enterprise\_linux/8/html/using\_selinux/index了解更多。}。对于使用SELinux模式创建的文件,在权限后面会有\code{.}作为标志(如同上面的\code{cd}命令一样)
\end{extend}

在文件权限中,每种用户又包含有\emph{可读权限}(\code{r}),\emph{可写权限}(\code{w})和\emph{可执行权限}(\code{x})。可读权限表明用户可以读取文件内容,对于目录而言表示用户可以查看目录内文件;可写权限表明用户可以修改文件内容,对于目录表示用户可以移动、删除目录内文件;可执行文件表明用户可以执行文件(一般为脚本文件或其他程序),对于目录表明用户可以进入目录。

\begin{attention}
    每种用户的权限一共就这三种,且数量和顺序都是固定的(即\code{rwx});对于没有的权限,使用短横线\code{-}表示无该权限。例如,\code{r-x}表示可读可执行,但不可写(如同上面的\code{/bin/cd}一样)。
\end{attention}

考虑一个很简单的例子:如果一个文件的权限表示为\code{-rw-rw-r--},对其他用户而言,含义是什么呢?

\answer{该文件对其他用户而言只读。}

\subsection{修改文件权限 \keyword{chmod}}\label{subsec:文件权限管理-修改文件权限}

在Linux当中,可以使用\code{chmod}修改文件的权限。修改方式有2种,第一种是直接设定三种用户的所有权限(共9位)。但是,如果直接写类似于\code{rw-rw-r--}这样子的形式的话,就稍显繁琐。如果考虑到特定的位数,可以把这三个权限的开关看作二进制的“0”和“1”,其中拥有权限为“1”。这样子就可以通过1个十进制数字表示一种用户的权限。例如,对于\code{r--},可以将其写作\code{100}也就是十进制的\code{4};类似地,对于\code{r-x},可以写作\code{101}也就是\code{5}。

\begin{extend}
    关于进制转换,对于任意一个$p$进制数,$\cdots a_2a_1a_0.a_{-1}\cdots$,其中$a_n,n\in\mathbb{Z}$表示在$0\sim p-1$范围内的数,将其转化为十进制的方法是
    \begin{equation*}
        \sum_ia_ip^i
    \end{equation*}
    例如,对于二进制数101,转化为十进制可以算作$1\times2^2+0\times2^1+1\times2^0=5$。对于十进制转二进制,可以使用\emph{短除法}进行,具体可以参考网站https://blog.csdn.net/weixin\_51472673/article/details/122482602。

    在有些专业教材中,可能会使用\code{0b}表示二进制数,例如\code{0b101},其中\code{b}为二进制单词binary的缩写。
\end{extend}

使用这种数字表示方法修改权限的格式为\code{chmod <权限> <文件路径>}。例如,对\code{supercell}文件执行\code{chmod 755 supercell}则表示最终的权限为\code{rwxr-xr-x}。

思考一下,如果想让一个目录只对所有者提供全部权限,而其他人无权限,用数字表示应当怎么写?

\answer{700表示\code{rwx------}}

虽然这种方法可以修改所有权限,但有时我们仅仅希望添加或删除某一特定的权限。例如,当我们编写了一个程序后,可能只希望给它添加一个对所有者的可执行权限,甚至不关心它对其他用户的权限如何,如果再一点点算,就有点麻烦。例如,原本的权限是\code{rw-r--r--},诚然使用\code{chmod 744 <文件路径名>}可以实现这一功能,但有没有更简单的方法呢?

在\code{chmod}命令中,除了使用数字表示权限外,还可以利用如\code{+}, \code{-}和\code{=}这样的符号进行增添、删除或修改权限。其基本用法为\code{chmod [用户]<操作符><权限> <文件路径>}。其中\code{[用户]}表示想给所有者(\code{u}),所属用户组(\code{g})还是其他用户(\code{o})修改权限。对于操作符而言,\code{+}, \code{-}和\code{=}分别表示增加权限、删除权限和更改权限。后面的\code{<权限>}使用\code{r}, \code{w}, \code{x}这种表示方法。

举个例子,如果我们希望给文件所有者添加一个可执行权限,则可以直接执行\code{chmod u+x <文件路径>}即可;如果希望给其他人删除可读权限,则使用\code{chmod o-r <文件路径>}。如果希望给所有者和所属组设置为可读写权限的话,可以用\code{chmod ug=rw <文件路径>}。

\begin{attention}
    正如后面的例子所展示的那样,\code{[用户]}和\code{<权限>}都可以一次性写多个。当然,有时可能会希望一次性给所有用户设置权限,例如给所有人可读写权限,在\code{chmod}当中,对于\code{[用户]}还提供了一个选项\code{a},表示\emph{所有人}。例如,\code{chmod a=rw <文件路径>}表示设置为所有人可读写。

    同时,利用这种方法也可以设置不同的权限,即一次性设置多个用户,其中用\code{,}分隔。例如,如果希望将权限设置为\code{rwxrw-r--},则可以直接写作\code{chmod u=rwx,g=rw-,o=r-- <文件路径>}
\end{attention}






\subsection{错误处理}\label{subsec:文件权限管理-错误处理}
% 请在本节列出可能遇见的错误与解决方法

\subsubsection{希望给文件所有者自己设置权限,使用数字之后为什么给其他用户设置权限了,而所有者没有权限}

在使用数字表示的时候,三位分别表示所有者、所属组和其他用户。对于不足三位的数字,其前面默认\emph{补零},例如,\code{chmod 7 <文件路径>}实际上等价于\code{chmod 007 <文件路径>}

\begin{extend}
    对于稍微了解计算机的读者来说,也许你已经有所察觉了。我们之前说这个数字是十进制数字,但严格来说,它是\emph{八进制}数字。对于八进制而言,它的每一个数位上的数字,都与3位二进制对应。例如,八进制的7表示二进制的111,八进制的3表示二进制的011等。具体转换如表\ref{tab:文件权限管理-十进制,二进制和八进制对应表}所示。

    \begin{table}[h]
        \centering
        \begin{tabular}{c|c|c||c|c|c}
            \hline
            \multicolumn{1}{c}{\textbf{十进制}} & \multicolumn{1}{c}{\textbf{二进制}} & \multicolumn{1}{c}{\textbf{八进制}} & \multicolumn{1}{c}{\textbf{十进制}} & \multicolumn{1}{c}{\textbf{二进制}} & \multicolumn{1}{c}{\textbf{八进制}}\\
            \hline
            0 & 000 & 0 & 4 & 100 & 4\\
            1 & 001 & 1 & 5 & 101 & 5\\
            2 & 010 & 2 & 6 & 110 & 6\\
            3 & 010 & 3 & 7 & 111 & 7\\
            \hline
        \end{tabular}
        \caption{十进制,二进制和八进制对应表}
        \label{tab:文件权限管理-十进制,二进制和八进制对应表}
    \end{table}

    在一些专业的计算机书籍或其他地方,八进制会用\code{0o}作为前缀(其中第一个是数字0,第二个是小写字母o)。例如,0o5=0b101=5。当然,因为八进制每一个数位的范围0到7小于十进制0到10的区间,因此对于7以内的数字而言,八进制和十进制是一样的。但随着数字增加,二者会出现差别,但八进制和二进制之间仍存在一一对应关系。例如,23=0b010111=0o27。

    事实上,每三位二进制的对应关系产生了八进制,而目前更常用的是四位二进制对应关系所产生的十六进制(前缀为\code{0x})。
\end{extend}