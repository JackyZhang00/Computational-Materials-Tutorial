\chapter*{写在前面的一些说明}

从本部分开始,就进入到了这一教程的主体部分--\emph{VASP计算}。在开始计算之前,有一些注意事项需要说明:

\section*{关于输入文件}

随着计算任务的不同,VASP所需要的输入文件也是不同的。对于\code{INCAR}和\code{KPOINTS}文件\footnote{这些文件的具体含义在后续教程中都会详细介绍。},通常是与具体的计算任务有关,且可以借助于如vaspkit的脚本生成。而对于\code{POSCAR}文件而言,其表示所要计算的结构信息,对于不同的课题组、不同的研究课题,所研究的结构也会千差万别。在教程中为了演示方便,有时会设定某一特定的结构作为\code{POSCAR}文件,仅作为演示用,在实际使用时需要根据具体问题设置不同的文件。

对于\code{POTCAR}文件而言,通常对于已经购买版权的课题组而言,都会有一个配套的\code{POTCAR}目录,里面会包含有所有元素的赝势文件。对于这种情况,通常使用如vaspkit的脚本生成并不是难事(同样也可以使用Linux命令手动生成,具体内容在后续章节会进行介绍)。对于没有购买版权的课题组来说,可以“暂时借用”别人已有的文件\emph{作为练习},但不能将其用于课题组的论文当中。

\section*{关于提交脚本}

之前所介绍的Linux命令,在大多数课题组的系统中都是可以使用的。但VASP却不是如此。首先,不同课题组的VASP版本可能不同,有时不同版本的命令或参数含义可能会有些许变化,但这种变化通常是影响较小的。最重要的是,由于所使用的系统环境不同,例如,对于本地运行和运算集群运行,其提交任务的方法可能会有些许差异。目前,大多数课题组在计算VASP任务时都是采用服务器集群进行计算,此时就会需要一个叫做\emph{排队系统}的东西。

对于不同课题组的不同集群,所使用的排队系统可能不同。本教程在编写时,通常是使用slurm作业管理系统,目前如中国科学技术大学、上海交通大学等学校的计算中心都是采用这一管理系统。对于使用其他管理系统的课题组而言,需要参考自己课题组的使用方法。

\section*{使用slurm的命令与方法}

考虑到教程的完整性,这一节简单介绍关于slurm的命令。

\begin{attention}
    这一部分仅仅适用于那些使用slurm作业管理系统的课题组,对于其他课题组,则需要参考自己课题组的使用方法。
\end{attention}

在使用slurm时,需要配合以一个提交任务脚本。一个典型的提交任务脚本\code{sub.vasp}如下所示:

\begin{lstlisting}[caption=sub.vasp]
#!/bin/bash
#SBATCH -n 56
#SBATCH -N 1

# 打印任务信息
echo "Starting job $SLURM_JOB_ID at " `date`
echo "SLURM_SUBMIT_DIR is $SLURM_SUBMIT_DIR"
echo "Running on nodes: $SLURM_NODELIST"

# 执行任务
## 载入vasp
module load VASP
ulimit -s unlimited
mpirun vasp_std > vasp.out 2>vasp.err

# 任务结束
echo "Job $SLURM_JOB_ID done at " `date`
\end{lstlisting}

其中,\code{\#SBATCH -n}表示\emph{任务所使用的核数},在本例中设定为56核;\code{\#SBATCH -N}表示\emph{使用的节点数}。在上述代码中,表示使用56核,1个节点进行计算。

而对于中间的命令,特别的如\code{ulimit -s unlimited}表示\emph{不设置内存限制};\code{mpirun vasp\_std > vasp.out 2>vasp.err}表示通过\code{mpirun}(即\emph{并行计算程序})运行\code{vasp\_std}命令并将输出结果保存至\code{vasp.out},而将错误信息输出到\code{vasp.err}当中。

\begin{extend}
    对于有显卡加速的课题组而言,可能需要在前面指定\code{\#SBATCH --partition=GPU}指定使用的GPU(如\code{a100}等),同时使用\code{\#SBATCH --gres=gpu:$n$}表示调用$n$张显卡进行计算。
\end{extend}

提交任务时,需要将提交脚本和计算目录放在一起,在计算目录下使用\keyword{sbatch}\code{ sub.vasp}\footnote{提交脚本名}即可。除此之外,slurm还有如下命令:

\begin{itemize}
    \item \keyword{squeue}表示查看当前任务队列
    \item \keyword{scancel [jobID]}表示取消\code{[jobID]}编号的任务,例如,\code{scancel 6066}表示取消编号为6066的任务;
\end{itemize}

除此之外,slurm也可以使用如\code{srun}或\code{salloc}提交交互式作业,或者申请特定的资源并登录至节点。这一部分内容在VASP计算时不会使用到,因此不在这里介绍。