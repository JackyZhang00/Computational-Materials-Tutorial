\chapter*{Python与机器学习:前言}

这一部分是整套教程的最后一部分,主要介绍关于机器学习的相关算法和程序实现方法。目前,机器学习以及人工智能相关算法与技术正在爆发式涌现出来,大预言模型(LLM)频繁更新迭代,在科研领域使用人工智能的相关技术正逐渐成为科研领域的前沿方法。尤其是2024年诺贝尔物理学奖和化学奖分别颁发给与人工智能相关的技术,更是进一步促进了人工智能在科研领域的应用。

得益于Python语言强大的库函数支持,目前大多数机器学习算法都是基于Python语言实现。考虑到读者的编程水平不同,本部分的前两章将首先介绍关于Python语言的基本应用,分别针对Python的内置语法(如基本的数据结构、程序结构等)和频繁使用的库函数(如NumPy、Matplotlib等)进行讲解。

之后,我们将详细介绍大多数机器学习算法的原理和具体应用,由于计算性能的限制,考虑到机器学习算法在性能不是太好的电脑上也可以运行并得到理想结果,我们会花费较大的篇幅介绍机器学习的算法,包括它们的数学原理、程序实现和具体的应用方法。

对于有条件的情况下,我们将简单介绍关于神经网络(在有些地方可能也会将其称作“深度学习”)的一些算法与实现。对于有能力(特别是算力)的读者可以尝试了解并编写一些简单的程序来运行它们。考虑到大多数读者所使用设备的算力水平以及所使用的硬件,本教程在深度学习部分所使用的代码大多是基于英伟达Nvidia的常规显卡(如Nvidia GTX4060等显卡)实现并测试的。因此,我们并不会涉及太大与太复杂的神经网络。为了拓展大家的应用场景,我们在最后会讨论使用云端服务器的算力(考虑到像Colab等服务器对网络要求较高,且不稳定,我们在介绍云服务器时会使用百度飞桨的云服务器进行演示)运行神经网络的方法。

\section*{关于练习题与难度分类}

这一部分的很大内容是在介绍编程的,因此不做编程练习是不行的。同时,在介绍机器学习算法时,我们又不得不介绍很多的数学原理以及相应的概念,因此,也需要读者对其中一些数学原理有所了解。在每一节的结束时,我们都会尽量设置一些练习题,以便读者可以练习并检测自己编程能力的掌握情况。但不同的难度放在一起,对于初学者而言做困难的题目显然是有些不合适的。因此,我们希望参考《计算机程序设计艺术》这本书中对练习题的难度分类算法,简单将题目难度的“十位数”划分为以下几个类型:

\begin{itemize}
    \item 00: 这种题目通常都是非常简单的概念题,对于大多数读者而言,这类题目在几秒钟之内就可以思考出来(但需要你对本节内容有所掌握)
    \item 10: 这种题目可能需要你稍作思考,甚至可能在必要的地方需要你回去查阅相关的内容,但难度不会太大。对于编程题而言,这通常会是10行左右的程序,你可能在3-5分钟时间就可以思考出来并完成它们;
    \item 20: 这种题目的思考量会明显更大一点,可能需要你对前文内容有足够的理解。对于编程题而言,这可能需要你稍微思考一下算法(可能并不简单),并且程序量大约在20行-30行左右。你可能需要十几分钟乃至几十分钟左右来思考并完成它们;
    \item 30: 对于程序算法的思考量,或者运算会更多。在这时候你可能会需要编写一个比较完整的程序(或者算法)来实现一个具体的功能。程序量大约在50行左右,且所使用的算法更复杂。这种题目通常适合作为程序设计课程的课后作业,你所需要的时间大约以小时计;
    \item 40: 这种题目可能在编程上更适合作为课程设计。程序量大约在100行左右,且需要你对算法、程序语言有更深的理解,甚至需要你有一些创造性的想法在里面。通常来说,完成这个题目可能需要以天为单位;
    \item 50: 这种题目往往是很难实现的,或者这种题目绝不是可以凭一己之力完成的。对于一些特殊的内容(尤其是在机器学习部分),如果你有幸实现了这种算法,请务必分享给我们。我们会非常高兴看到这一进步。
\end{itemize}

在题目编号中,为将“编程题”和一般的概念题做区分,我们会使用“P”来表示编程题。同时,在机器学习部分,有些题目可能对于数学要求较高(可能使用了高等数学部分的内容),我们会使用“M”表示。

题目难度数值对5的余数表示所需要的时间量估计,对于再一步细分的难度,则在中间插值取得,将其划分为分度值为1的刻度。因此,对于难度为25和难度为24的题目,前者所需要的时间可能更少,但需要更多的思考量和创造性。

这里特别注意的一点是:对于出题的人来说,严格给出题目难度是非常困难的,而且所使用的程序行数也不是绝对的。因此,每个题目的难度仅是作为读者选择题目的参考(我们强烈建议读者完成难度在30以下的所有题目)。

\section*{关于这一部分的“错误处理”}

对于编程而言,我们所说的“错误”可以包括两种类型——一种是编程语言语法上的错误,另一种是算法上的错误。在每一节的后面,我们会尽可能给出代表性的编程时所可能发生的错误,这些错误很多都是“致命性”的,即程序语言会中断执行的错误(在后面我们会了解到,这种东西叫做“异常”)。这些错误在很多时候是可以避免的。

而对于算法上的错误,我们确实很难避免,而且这些错误难以排查。它们在程序编写过程中,可能是一个符号输入错误,或者变量赋值错误等。这些错误导致程序在程序层面是正确执行了,但没有得到预期的结果。这一部分错误我们只能尽可能在正文中进行“启发”,从而帮助读者理解算法设计时需要注意的一些事项。对于笔误等原因所发生的错误,我们“心有余而力不足”。

\section*{练习题}

\begin{itemize}
    \item [00] 这一部分我们所需要学习的编程语言是什么?
    \item [M14] 对于函数$f(x)=x\sin(2x)$,求它的导函数$f_x(x)$
    \item [P50] 编写一个大预言模型(LLM),帮助你回答上面的问题。
\end{itemize}

