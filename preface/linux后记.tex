\chapter*{Linux基础:后记}

到这里,关于Linux基础部分的所有章节就全部结束了。如果已经完全通读了这一部分的所有内容,那至少在Linux操作上应该是没有太多问题了。下面,让我们再次回顾一下这一部分的所有内容,确认一下你是否已经掌握了它们:

在第一章,我们第一次接触了Linux操作系统,我们学习了如何使用命令行的命令对目录和文件进行简单的操作——例如,\emph{复制、粘贴、删除};我们学习了怎样进入一个目录(如同在Windows下双击鼠标的效果),学习了怎样查看并修改一个文件的权限,从而保护自己的文件不被修改(以及更常见的,为后续脚本提供可执行权限);我们学习了怎样压缩一个目录,并可以利用这个方法方便地下载与上传文件。

在第二章,我们进一步了解了关于文本编辑的部分。类似于Windows下的记事本,在Linux中,我们介绍了\code{vi}的操作:包括创建一个文本文件,以及利用vi的一些内置命令查找或替换一些特定的字符串。在最后,我们甚至介绍了一些关于正则表达式的相关内容,利用它们可以方便我们查找具有\emph{固定模式但内容不同}的字符串。

从第三章开始,我们就逐渐开始了“偷懒”的行为。为了解决\emph{大批量处理的问题},我们尝试使用命令行的命令如\code{sed}和\code{grep}对文本文件进行替换和查找。利用这种方法,并配合有一些像\code{for}循环、通配符、管道运算符等方法,可以批量处理文件与目录。例如,我们可以利用\code{for}循环,一次性创建目录名为1-10的目录。

对于经常使用的一些命令,我们可以尝试将其写成一个\emph{脚本},并通过执行脚本运行特定的命令序列。在第四章,我们详细介绍了如何创建一个脚本程序,包括基本的程序输入和输出,并且进行简单的运算;对于一些复杂的情况,我们也学习了利用\code{if}等类似语句进行条件分支判断,利用\code{for}和\code{while}等类似语句进行循环,并可以对其进行控制。对于较大体量的脚本程序,我们也介绍了类似于\emph{函数}和\emph{数组}的方法,可以帮助你更方便地管理和编写程序(这两部分是可选阅读部分)。

这一部分与其他部分(尤其是第二部分关于VASP的部分)不同的是,在这一部分所介绍的大多数内容都是前后连续的,也正如你在第四章所感受到的那样,程序量会逐渐上升、程序语法难度和算法难度也逐渐加大。也许,前面的一部分没有掌握,会极大影响后面内容的理解。但从另一个方面看,Linux终究是一个“工具”,它并不应该成为你做科学研究的障碍。正因如此,我们在编写这一部分时,并没有加入太多更复杂的内容,例如服务器维护、操作系统基础和账号管理相关的内容。因为对于大多数科研人员而言,Linux操作系统只是计算的一个平台,并不是科研的全部。因此,这一部分——重要,但不足够重要。

在后续的使用过程中,你必将会面临一些问题(可能是没注意的细节、或者是遗忘的命令或参数),这都没有关系。只要记得:\emph{随时回来查找},把这一部分当成“工具书”即可。

最后,在创作这一部分的章节时,由于本人水平有限,可能或多或少有一些错误,还是恳请大家对其中的错误进行指正。我也自知Linux系统的部分远不止此,但碍于个人水平,可能还是有很多细节(例如命令的使用和参数)没有介绍到。如果有目前没有介绍到的细节,也希望大家指出来,在进一步了解后会进行增补修正。

在创作过程中,可能或多或少参考引用了“菜鸟教程网站”、“CSDN”网站、“知乎”和《鸟哥的Linux私房菜》参考书的相关内容,在文中没有进行详细标注,在此对以上参考资料中相关的创作者致以谢意。