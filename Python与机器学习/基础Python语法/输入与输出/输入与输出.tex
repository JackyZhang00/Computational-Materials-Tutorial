% 请在下方的大括号相应位置填写正确的节标题和标签,以及作者姓名
\section{输入与输出}\label{sec:输入与输出}
\sectionAuthor{Jiaqi Z.}

% 请在下方的item内填写本节知识点
\begin{Abstract}
    \item 什么是标识符及其命名规则
    \item 如何使用\code{input()}函数输入
    \item 如何使用\code{print()}函数输出
    \item 变量与赋值
\end{Abstract}

在本节,我们将正式开始Python程序设计。与其他程序设计语言类似,我们首先需要了解基本的输入和输出方法,这也是我们后续进行程序交互的基础。

\subsection{标识符}\label{subsec:输入与输出-标识符}

在编程语言中,标识符用来标记某个实体,例如变量、函数等都需要一个名字来表示它们,这个“名字”就叫做\emph{标识符}。与大多数编程语言类似,Python语言的标识符有如下要求:

\begin{itemize}
    \item 只能包括数字、字母和下划线;
    \item 第一个字符不能是数字;
    \item 不能以关键字作为标识符。
\end{itemize}

\begin{attention}
    在Python中对大小写敏感,因此,标识符\code{Apple}与标识符\code{apple}是不同的标识符。

    所谓关键字,指的是那些已经被Python语法所采用,表示特殊含义的单词(如\code{if}、\code{for}、\code{def}等),我们将在后面的学习中依次见到它们,此外,对于使用IDLE以及一些常见的开发环境,当使用关键字时会有特殊的字体显示(例如在IDLE当中,关键字默认会使用橙色)

    虽然Python标识符允许使用下划线作为开头,但在Python中,下划线开头往往具有特殊的含义,例如,使用一个下划线开头(例如,\code{\_name}表示\emph{私有的对象}(将在面向对象编程介绍时详细说明这一点)。其他具体的使用将会在后面说明。
\end{attention}

\begin{extend}
    在Python2当中由于使用ASCII编码,因此,Python2不能使用其他字符作为标识符。而在Python3中将字符集进行了扩充(PEP 3131提案\footnote{详见网站:https://peps.python.org/pep-3131/}),因此可以使用中文作为标识符,例如,我们可以将变量命名为\code{变量}。但我们强烈不建议这样做,会引起不必要的麻烦。
\end{extend}

\subsection{变量}\label{subsec:输入与输出-变量}

在前面我们已经或多或少提到了\emph{变量}这个词。在计算机中,变量指的是\emph{随着程序运行可以发生的量},相对地,\emph{常量}指的是\emph{不变的量},例如,一个明确的数字如\code{3.14}就是一个常量。

正如前面在介绍标识符所说的那样,在Python当中使用标识符标记变量。给变量一个值的方式是使用\emph{赋值}运算符\code{=},例如,\code{a=5}表示将变量\code{a}的值设置为\code{5},或者说把\code{5}赋值给\code{a}。

类似地,赋值运算左右两端可以同时是变量,例如,\code{b=a}表示将\code{a}的值\emph{赋值给}变量\code{b}。

\begin{attention}
    在前面的Linux教程中我们已经提到了关于赋值运算符的注意事项,但考虑到读者可能没有看到前面的内容,因此我们这里再提一下:

    在上面我们提到\code{b=a}的含义是\emph{将\code{a}的值赋值给\code{b}},这对于了解过其他编程语言的读者而言是很自然的。但如果你没有学习过其他编程语言,需要特别注意的一点是:在程序设计语言中(几乎大多数的程序语言),\code{=}所表示的含义与你所熟悉的数学上的含义不同。在数学上,=表示一种\emph{状态},表达两个值相等;而在程序设计中,\code{=}表示将右边的值\emph{赋值给}左边的值(一种\emph{动作})。

    尽管在最后的结果上,二者是相同的,但数学上的=表达一种“状态”,而程序设计中表达一种“动作”,是不同的。一个很简单的例子就是上面的\code{b=a},从数学的角度看,这显然不成立,因为在开始时两个变量的值并不一定是相等的(大概率不相等),但程序设计上是可行的,因为它表达了“赋值”的动作。

    也正因如此,在数学上,$a$和$b$相等写成$a=b$或者$b=a$都是可行的(这也就是等式的“对称性”);而对于程序设计而言,\code{a=b}和\code{b=a}显然是不同的,因为它们所表达的动作“方向”是不同的。
\end{attention}

\subsection{\code{print}输出}\label{subsec:输入与输出-print输出}

在\ref{sec:安装Python}一节的后面,我们已经使用了\code{print()}函数实现了输出。虽然我们之前是对一个\emph{字符串}进行输出,但这个函数也可以对任意变量进行输出。例如,下面的代码实现了一个很简单的功能,测试变量赋值与变量输出:

\begin{lstlisting}[language=python,caption={simple\_program.py},numbers=left]
# simple_program.py
# 一个简单的输出
# 变量赋值
a=5
b=6
# 输出变量
print(a)
print(b)
# 变量赋值
b=a
# 使用print()函数输出多个变量
print(a,b)
\end{lstlisting}

在这段代码中,我们使用\code{\#}表示\emph{注释}。与其他编程语言类似,在Python中注释不会被解释执行,你可以使用任何语言来书写注释。

代码的第4行和第5行分别将\code{5}和\code{6}赋值给变量\code{a}和变量\code{b},然后使用\code{print()}函数依次输出了这两个变量的值。

在调用\code{print()}函数时,括号内直接给出要输出的内容即可。除了变量之外,括号内也可以提供常量,例如,\code{print(5)}可以直接输出\code{5}。

\begin{attention}
    在后面的章节我们会介绍到,像\code{print()}这种需要使用一个括号调用的命令叫做\emph{函数}。一个函数主要由三部分组成:函数名、参数和返回值。在这里的\code{print}就是\emph{函数名},括号里的内容就是\emph{参数}。

    对于\code{print()}函数而言,是没有\emph{返回值}的。而有些函数是可以返回一个值并参与后续运算处理的,例如后面要介绍的\code{input()}函数就是这样。

    后面我们还会详细介绍关于函数的细节。
\end{attention}

程序第10行对变量进行了\emph{赋值},正如前面所说的那样,这里的\emph{=}不表示数学上的“相等”关系(从数学的角度来看,5和6显然不相等),而是表示\emph{将\code{a}的值赋值给\code{b}}。经过这一操作后,\code{b}的值也变成了\code{5}(\code{a}的值)。

最后,程序在12行输出了两个变量的值。特别的,这里的\code{print()}函数括号内使用逗号分隔两个变量。对于\code{print()}函数而言,括号内的多个参数可以使用逗号分隔,程序会依次输出括号里的参数,在输出时使用空格分隔输出结果。

\begin{attention}
    在这里我们也提前展示了Python里面函数参数的一种用法:\emph{Python在调用函数时多个参数在括号里使用逗号分隔}。
\end{attention}

\begin{extend}
    我们说的“输出时使用空格分隔”,这是\code{print()}函数默认的设置。在实际使用时,可以通过设置参数进行修改。例如,希望每个参数使用“-”符号分隔,则可以使用\code{print(a,b,sep="-")}。其中\code{sep=}表示\emph{关键字参数}。在后面介绍函数时我们会详细说明这些。

    此外,\code{print()}函数默认输出后会进行换行(即\code{$\backslash$n}),这个也可以修改。与上面的方法类似,可以使用\code{end}关键字。例如,在上面的基础上如果希望使用空格做结尾,则可以写作:\code{print(a,b,sep="-",end=" ")}
\end{extend}

结合上一节的“Hello World!”程序,在输出时也可以将常量和变量放在一起输出。例如,\code{print("a=",a)}将会在屏幕上输出\code{a=5}。

\subsection{\code{input}输入}\label{subsec:输入与输出-input输入}

任何一个程序,都应当具有输入和输出功能。这里的输入和输出可以是多种形式,例如,输出可以是在命令行的输出(类似于前面所说的\code{print}函数),也可以是文件的输出,或者控制硬件输出(例如通过控制七段数码管的不同引脚实现一位数字的输出、或者直接控制打印机实现文字输出)。对于输入,也可以是多种形式,例如我们这里要介绍的通过命令行输入,或者文件输入,甚至更新的语音输入等。

在Python中是通过\code{input()}函数实现\emph{字符串}的输入。正如前面所说的那样,输入函数是具有返回值的,其返回的内容就是\emph{输入的字符串}。例如,下面这段代码实现了一个最简单的“应声虫”功能(程序输出用户输入的内容)。

\begin{lstlisting}[language=python,caption={my\_echo.py},numbers=left]
# my_echo.py
# 输出用户输入的字符串
a = input("请输入字符串:")
print(a)
\end{lstlisting}

\begin{attention}
    \code{input()}函数与\code{print()}函数类似,也可以提供\emph{参数},其含义表示需要在命令行显示的“提示内容”。以上面的程序为例,程序在等待输入的时候,会输出“请输入字符串:”作为输入提示。
\end{attention}

在上述代码的第3行,同时进行了\code{input()}函数的输入和变量的赋值。在执行这段代码时,程序会首先输出“请输入字符串:”这样的提示信息,然后会停下等待用户输入。当你输入一些内容后,使用回车表示输入结束,程序会读取到你输入的内容并将其赋值给变量\code{a}。然后在第4行实现了变量的输出(也就是所输入字符串的输出)。

\begin{extend}
    对Python2而言,\code{input()}函数需要输入的是一段\emph{表达式},它会自动识别对应的类型。例如,当用户输入\code{123}时,程序会自动识别为整数;而输入\code{"Hello"}时,程序会识别为字符串。

    看似这是一件方便的事情,但当用户输入字符串时需要特别输入前后的引号表示字符串,如果用户忽略引号而直接输入\code{Hello},程序会报错。

    在Python3当中,用户输入的一切内容都会认为是\emph{字符串}。
\end{extend}

\subsubsection{*使用\code{input()}函数读取多个数据}

这一部分内容是后面章节需要介绍的,但考虑到后续练习题可能会使用到,我们简单介绍一下关于\code{split()}函数(或者更准确的说,叫“成员函数”\footnote{这是一个在面向对象编程语言中针对对象所使用的术语})在读取数据时的应用。

与\code{print()}函数和\code{input()}函数直接使用不同,\code{split()}函数需要在前面指定一个特定的字符串(对象),并在后面使用\code{.}运算符调用函数。例如,我们希望对字符串"Hello World!"使用这个函数,则需要写作\code{"Hello World".split()}。这个函数的作用是将字符串分成若干部分(划分的分隔符可以在参数中指定,不指定时默认为空格分隔),调用后函数会返回一个“列表”\footnote{后面章节会介绍到的一种有序序列类型。},例如,上面的语句将返回\code{["Hello","World"]}列表。

\begin{attention}
    使用\code{split()}划分的字符串,是不带分隔符(空格)的。
\end{attention}

\begin{extend}
    后面的章节会详细介绍列表,但这里我们已经自然而然给出了列表的“表示方法”,即\emph{使用中括号,且里面每个元素用逗号分隔}。
\end{extend}

在使用\code{input()}读取多个数据时,可以输入使用空格或其他分隔符分隔的数据,使用\code{split()}函数类似上面的方法对输入的字符串进行划分。在变量赋值时,可以直接赋值多个变量(前面变量名用逗号分隔)。还是以前面的例子为例,赋值时可以使用语句\code{a,b="Hello World".split()}而将变量\code{a}赋值为"Hello",变量\code{b}赋值为"World"。

\subsection{练习题}\label{subsec:输入与输出-练习题}

\begin{itemize}
    \item [10] 如果有两个变量\code{a}和\code{b},请设计一种方法,将其变量的值交换(注意:语句需要按顺序执行,在必要时可能需要引入另外的变量)
    \item [P04] 编写一个程序,用户输入一个句子,输出这个句子三遍,中间使用分号分隔,分号前后没有空格。
    \item [P18] 日期的格式分为ISO格式和美式格式,其中ISO格式为使用"-"分隔,且日期顺序为"YYYY-MM-DD",例如,2025年3月23日表示为"2025-03-23";美式格式为使用"/"分隔,且顺序为"MM/DD/YYYY",例如,上面的日期表示为"03/23/2025"。同时在一些编码中(例如身份证或者电子表格数据处理中),可能没有分隔符而使用简单的八位数字排列(例如20250323)。请编写一个程序,用户输入ISO格式的日期,输出美式格式和数字编码格式的对应日期。为了程序设计简单,用户输入的月和日一定是两位数(不足时前位补0)。
\end{itemize}