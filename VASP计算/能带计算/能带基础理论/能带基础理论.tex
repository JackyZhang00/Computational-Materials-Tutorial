% 请在下方的大括号相应位置填写正确的节标题和标签,以及作者姓名
\section{能带基础理论}\label{sec:能带基础理论}
\sectionAuthor{Jiaqi Z.}

% 请在下方的item内填写本节知识点
\begin{Abstract}
    \item 什么是能带
    \item 能带的三个重要近似
\end{Abstract}

在本章,我们将要了解材料计算的一个重要内容--关于能带的计算。毫不夸张地说,能带论是目前研究固体中的电子状态,说明固体性质最重要的理论基础。一个最简单的例子是,利用能带的相关计算,我们可以从严格的角度判断材料的导电性。

\begin{extend}
    导体和绝缘体的概念,贯穿了我们的学习生涯。而这一概念也在随着认知水平的增长发生变化。

    最开始接触的时候,我们认为,导体是那些带有电荷的物质,而绝缘体内部没有电荷。这一结论是我们对“电”和“电荷”的初步认识,显然是不准确的。进一步学习了物理后,我们了解到:任何原子都是由原子核和电子组成的,所谓的导体,就是存在“自由移动的电子”,相反,绝缘体就是没有自由电子的物质。这一概念结合了原子和电子的认识,相比于最开始的“电荷”,显然更准确了。

    到了现在,我们将要了解到能带。基于能带理论,在导体中,价带(价电子所在的能带)和导带(电子可以自由移动的能带)是重叠的,或者价带顶部和导带底部之间的能隙(带隙)非常小,甚至为零。这意味着电子可以轻易地从价带跃迁到导带,从而在电场作用下自由移动,形成电流。而对于半导体而言,价带和导带之间存在一个较大的带隙,电子要跃迁到导带需要吸收足够的能量(如热能或光能)。在常温下,电子通常没有足够的能量来跃迁,因此电子不能自由移动,导致绝缘体不导电。
\end{extend}

\subsection{什么是能带}\label{subsec:能带理论基础-什么是能带}

从原子物理的知识来看,一个孤立原子的电子只能处在特定的能级当中。而我们所计算的材料,往往是周期性的多原子材料\footnote{对于VASP而言,所有材料都是“周期性”的。我们所说的单原子或单分子计算,通常是通过调整晶格的大小,从而减弱相互之间的作用,近似成“单分子”计算。}。对于多原子而言,电子和电子之间、电子和原子核之间会产生相互作用,此时电子的能级就会发生“展宽”,从而变成一系列的带状结构。称为“能带”。

\begin{attention}
    通常情况下,能带仅在周期性材料中讨论。对于单分子(如气体分子等),计算能带往往没有意义。
\end{attention}

% \subsubsection{小小节标题}

\subsection{能带理论的三个近似}\label{subsec:能带理论基础-能带理论的三个近似}

我们不会在这一教程中详细介绍能带的推导过程,但是我们还是有必要在这里提到三个重要的近似。这些近似可以说是能带理论的基础,甚至可以说是整个第一性原理计算的基础。

\begin{itemize}
    \item Born-Oppenheimer近似,又名绝热近似:因为原子核比电子重的多,所以原子核比电子具有更大的惯性,更难运动。因此,我们只考虑电子的运动,原子核是被固定住的。
    \item 单电子近似(独立电子近似、平均场近似):将电子与电子相互作用等效成一个平均值,电子是在一个平均场中运动。
    \item 周期场近似:平均场是周期性的。
\end{itemize}

具体内容在任何一本固体物理教材中都会详细提到,这里不再赘述。