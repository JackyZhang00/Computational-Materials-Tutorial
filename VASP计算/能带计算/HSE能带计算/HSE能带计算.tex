% 请在下方的大括号相应位置填写正确的节标题和标签,以及作者姓名
\section{HSE能带计算}\label{sec:HSE能带计算}
\sectionAuthor{Jiaqi Z.}

% 请在下方的item内填写本节知识点
\begin{Abstract}
    \item 如何计算HSE能带
\end{Abstract}

% 请在正文相应位置填写正确的小节标题(或小小节标题),同时将标签的“节标题”和“小节标题”改为实际内容

在之前介绍PBE能带计算时提到,\emph{使用PBE计算会得到带隙偏小的情况},而这对于高精度计算显然是不合适的。因此,在大多数计算能带的论文中都需要其他泛函如HSE计算得到的能带。因此,在本节我们将讨论如何使用HSE计算能带并给出更加准确的带隙。

\subsection{结构优化与自洽计算}\label{subsec:HSE能带计算-结构优化与自洽计算}

在本节我们将讨论的结构是金刚石,其\code{POSCAR}文件如下所示:

\begin{lstlisting}[caption=POSCAR]
C2
1.0
        2.5269944668         0.0000000000         0.0000000000
        1.2634972334         2.1884414035         0.0000000000
        1.2634972334         0.7294804678         2.0632823422
    C
    2
Direct
        0.750000000         0.750000000         0.750000000
        0.500000000         0.500000000         0.500000000
\end{lstlisting}

对于HSE泛函计算能带,结构优化过程与前面\ref{subsec:VASP计算能带过程-结构优化}所介绍的步骤完全相同,因此这里不再赘述。

对于自洽计算,\code{KPOINTS}则需要使用hse的K点。生成方法可以使用\code{vaspkit-3-302/303}生成\code{KPATH.in}后使用\code{vaspkit-25-251}生成\code{KPOINTS},其中可以根据需要选择Monkhorst-Pack Scheme或Gamma Scheme方法生成K点\footnote{一般来说,使用Monkhorst-Pack生成K点精度较高,但对于六方晶系,则需要选择Gamma方法。}。然后需要依次输入scf所使用k点密度(通常设定为\code{0.04})以及计算能带所需要的k点密度(通常也设定为\code{0.04})

同时,在计算scf时还需要将\code{INCAR}的\code{LWAVE = .TRUE.}以生成波函数文件,以便计算能带时使用。下面是一个参考使用的\code{INCAR}文件:

\begin{lstlisting}[caption=INCAR]
Global Parameters
ISTART =  1
ISPIN  =  1
LREAL  = .FALSE.
ENCUT  =  600   
PREC   =  Accurate
LWAVE  = .TRUE.  
LCHARG = .FALSE.  
ADDGRID= .TRUE.

Static Calculation
ISMEAR =  0       
SIGMA  =  0.05    
LORBIT =  11      
NEDOS  =  2001    
NELM   =  60      
EDIFF  =  1E-08       
\end{lstlisting}

\subsection{HSE能带计算}\label{subsec:HSE能带计算-HSE能带计算}

上一步计算完成后,将\code{KPOINTS}, \code{POTCAR}, \code{POSCAR}和\code{WAVECAR}文件拷贝(或移动)至新的目录(暂且命名为\code{hse-band},使用\code{vaspkit-101-STH6}生成HSE计算的\code{INCAR}文件,调整必要的参数后如下所示:

\begin{lstlisting}[caption=INCAR]
Global Parameters
ISTART =  1
ISPIN  =  1
LREAL  = .FALSE.
ENCUT  =  600
PREC   =  Accurate
LWAVE  = .FALSE.
LCHARG = .FALSE.
ADDGRID= .TRUE.
    
Static Calculation
ISMEAR =  0
SIGMA  =  0.05
LORBIT =  11
NEDOS  =  2001
NELM   =  60
EDIFF  =  1E-08
    
HSE06 Calculation
LHFCALC= .TRUE.
AEXX   =  0.25
HFSCREEN= 0.2
ALGO   =  ALL
TIME   =  0.4
PRECFOCK= N
\end{lstlisting}

这里面的关键参数是\keyword{AEXX},表示杂化泛函所占的比重,通常设定为0.25是比较合适的。但\emph{这一比值会影响到带隙的大小},因此在必要的时候需要进行调整以适应实验结果。

将上述文件提交计算,由于HSE能带计算精度较高,因此需要时间较长\footnote{在测试计算时,64核计算时间为7.7个小时。}。

\subsection{HSE能带计算后处理}\label{subsec:HSE能带计算-HSE能带计算后处理}

与\ref{sec:能带绘图与后处理}一节所介绍的类似,在HSE计算后也需要对数据进行处理,绘图。但在一些细节上有些许不同:

\begin{itemize}
    \item 在绘制能带图之前,需要确保当前目录下有\code{KPATH.in}文件(可以直接使用\code{vaspkit-3-302/303}的方式生成,也可以直接从之前\code{scf}目录下复制过来;
    \item 使用\code{vaspkit-25-252}生成能带数据,所得到的\code{BAND.dat}和\code{KLINE.dat}文件可以将其放入Origin等软件中绘图(方法参考\ref{sec:能带绘图与后处理}一节,在此略)
\end{itemize}

生成能带数据后,可以通过\code{BAND\_GAP}文件查看计算得到的带隙。在本地测试时,得到结果为5.29 eV,与文献\footnote{Stoliaroff, A. \& Latouche, C. Accurate ab initio calculations on various PV-based materials: Which functional to be used? The Journal of Physical Chemistry C 124, 8467–8478 (2020).}结果5.38 eV接近。

同时,我们也可以利用\ref{sec:VASP计算能带过程}所介绍的PBE能带计算方法,对金刚石的PBE带隙进行计算。结果发现其带隙为4.12 eV,显然小于HSE带隙。
