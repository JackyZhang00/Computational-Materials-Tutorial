% 请在下方的大括号相应位置填写正确的节标题和标签,以及作者姓名
\section{固定某些原子的优化}\label{sec:固定某些原子的优化}
\sectionAuthor{Jiaqi Z.}

% 请在下方的item内填写本节知识点
\begin{Abstract}
    \item 如何在优化过程中固定某些原子
\end{Abstract}

在对某些结构进行优化时,我们可能希望仅优化个别原子以加速优化过程。例如,对于气体吸附的过程,在没有确定最终吸附位置时,我们可能会希望先只优化气体分子的位置而将下方的衬底作为“背景”固定,从而找到一个近似合理的位置。在本节,我们将讨论这一类问题的实现方法。

\subsection{如何固定原子坐标}\label{subsec:固定某些原子的优化-如何固定原子坐标}

无论是\code{ISIF=2}或者\code{ISIF=3},优化过程都包括对原子坐标的优化。如果希望固定某个原子的坐标,则需要按照如下步骤操作:

\begin{enumerate}
    \item 在\code{POSCAR}的原子个数下面一行添加\code{Selective dynamics}(只要保证首字母为S或s即可);
    \item 在下方每个原子坐标后面使用三个逻辑值(\code{T}或\code{F})表示是否对这一坐标分量优化(或固定)(中间使用空格分隔)
\end{enumerate}

还是以前面\ref{sec:对坐标进行优化ISIF=2}一节中关于\ch{O2}分子的优化例子,如果我们希望固定某一个氧原子,而让另一个氧原子只在$z$方向优化,则可以将\code{POSCAR}写成下面的样子:

\begin{lstlisting}[caption=POSCAR,numbers=left]
O2
1.0
5.0 0.0 0.0
0.0 5.0 0.0
0.0 0.0 5.0
O
2
Selective dynamics
Cartesian
0 0 0 F F F
0 0 1.2075 F F T
\end{lstlisting}

其中,第10行使用三个\code{F}表示三个坐标分量\emph{均不进行优化},第111行使用\code{F}和\code{T}表示$x$和$y$坐标分量\emph{均不进行优化},而仅优化$z$坐标分量\emph{优化}。

使用的\code{INCAR}和\code{KPOINTS}与\ref{sec:对坐标进行优化ISIF=2}相同,这里仅列举如下而不做解释(方便读者直接阅读本节):

\begin{lstlisting}[caption=INCAR]
ENCUT  =  600
LWAVE  = .FALSE.
LCHARG = .FALSE.
PREC   = Accurate
NSW    =  300
ISMEAR =  0
SIGMA  =  0.05
IBRION =  2
ISIF   =  2
EDIFFG = -0.01
\end{lstlisting}

\begin{lstlisting}[caption=KPOINTS]
K-Spacing Value to Generate K-Mesh: 0.000
0
Gamma
    1   1   1
0.0  0.0  0.0
\end{lstlisting}

优化得到的\code{CONTCAR}如下:

\begin{lstlisting}[caption=CONTCAR]
O2                                      
1.00000000000000     
    5.0000000000000000    0.0000000000000000    0.0000000000000000
    0.0000000000000000    5.0000000000000000    0.0000000000000000
    0.0000000000000000    0.0000000000000000    5.0000000000000000
O 
    2
Selective dynamics
Direct
0.0000000000000000  0.0000000000000000  0.0000000000000000   F   F   F
0.0000000000000000  0.0000000000000000  0.2476911285012822   F   F   T

0.00000000E+00  0.00000000E+00  0.00000000E+00
0.00000000E+00  0.00000000E+00  0.00000000E+00
\end{lstlisting}

其中,氧原子的键长变为1.2385 Å,与之前所优化得到的键长(1.23828 Å)接近。同时需要特别注意的是:优化后第一个原子\emph{仍然在原点位置},而第二个原子\emph{仅$z$坐标发生变化},从而实现了\emph{固定某些原子坐标}的目的。