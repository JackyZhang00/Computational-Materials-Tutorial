% 请在下方的大括号相应位置填写正确的节标题和标签,以及作者姓名
\section{为什么要进行结构优化}\label{sec:为什么要进行结构优化}
\sectionAuthor{Jiaqi Z.}

% 请在下方的item内填写本节知识点
\begin{Abstract}
    \item 为什么要进行结构优化
    \item 结构优化的原理
\end{Abstract}

在计算科学领域,一个广为流传的话是“Garbage In, Garbage Out!”,即\emph{错误的输入会得到错误的结果}。因此,对材料计算而言,在进行任何材料计算之前,都需要保证输入的正确性,包括\emph{参数设置}的正确以及\emph{结构}的正确。其中,参数设置的正确性将取决于研究问题和研究经验,结构的正确则首先取决于\emph{结构优化}。本章我们将介绍关于结构优化的一些设置方法与技巧。

\begin{attention}
    在任何时候,开始一个新的体系进行计算,都需要进行本章所介绍的结构优化,哪怕这个材料是实验测得的或者数据库所给的。
\end{attention}

\subsection{为什么要进行结构优化}\label{subsec:为什么要进行结构优化-为什么要进行结构优化}

结构优化是指对整个输入体系的坐标进行调整,得到一个相对稳定的基态结构。例如,在计算材料的吸附性质时,我们需要通过结构优化得到一个“稳定”的吸附结构,从而测得吸附距离和其他稳定时的性质。如果需要确认材料的稳定性(例如计算声子谱),则显然需要首先进行结构优化以确保结构的“稳定”(甚至需要更高的精度)。

也许你会有一个疑问:\emph{如果结构已经实验测得,还需要结构优化吗}?答案是\emph{需要},这是因为计算软件本身是考虑了许多近似情况,有时实验测得的结构和计算所使用的稳定结构有“细微”的区别。例如,对于氧气分子而言,使用数据库(CCCBDB)内的数据,其键长为1.2075 Å. 当我们使用不同的近似(考虑不同类型的赝势)进行结构优化\footnote{这里我们仅作为示例给出结论,并不提供完整的输入文件。},得到的键长是不同的。使用O和O\_s的赝势,在设置截断能为600 eV的情况下,计算得到的键长分别为1.23828 Å和1.29688 Å.

上面的键长变化可以极好说明对于不同的参数设置以及近似条件,所得到的结构可能是不同的。因此,哪怕是实验测得的或者数据库所给的,在开始计算时也需要进行结构优化。

\subsection{结构优化原理}\label{subsec:为什么要进行结构优化-结构优化原理}

一个完整的结构优化计算,可以分为\emph{电子弛豫}和\emph{离子弛豫}两部分。首先,程序会计算许多步电子弛豫,以确定当前结构下稳定的波函数。基于密度泛函理论的基本原理,利用波函数可以得到体系的能量,判断能量是否达到极值点(大多数情况是判断原子平均受力);如果原子平均受力小于某一个设定的值,则认为结构已经“稳定”,否则会按照一定的算法对原子坐标进行移动,得到一个新的结构,再重复上述步骤,直到寻找到“稳定”结构。

\begin{attention}
    上面我们所说的“稳定”,指的是能量达到一个“极值点”。而利用基本的数学与物理相关知识可以知道,真正的稳定结构(基态)应当对应结构能量的“最小值”,而寻找到的“极值点”不一定是最小值。

    上述情况则会导致在有些情况下结构优化会得到一个“非最小值的极值点”,因此在实际计算时,可能还需要使用不同的初始结构进行结构优化,并通过能量比较确定最终的结构。一个典型的例子是计算吸附时需要考虑多个吸附位点,优化得到不同的结构,并通过能量比较确定最终的稳定吸附位点。
\end{attention}
