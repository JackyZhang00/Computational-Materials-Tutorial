% 请在下方的大括号相应位置填写正确的节标题和标签,以及作者姓名
\section{一些结构优化的策略与加速方法}\label{sec:一些结构优化的策略与加速方法}
\sectionAuthor{Jiaqi Z.}

% 请在下方的item内填写本节知识点
\begin{Abstract}
    \item 为什么要进行分步优化
    \item 一些分步优化时需要注意的参数与方法
\end{Abstract}

如果你所研究的材料是很简单的材料,甚至是数据库或实验已经测得的,当你用这一结构进行结构优化时,其速度往往是很快的(至少相较于后面所要讨论的那些而言)但如果这个结构是不确定的,例如,你需要研究一种经过某种方式调控的材料,或者是两种组分相互作用形成的体系(例如气体吸附),现成的数据往往不足以构建一个完全符合实际的结构。因此,你需要通过\emph{结构优化}这一步寻找合理的体系结构并进行后续的计算。

但是,一个初始不合理的结构,其优化时长通常可能以小时或天计,而且其优化结果不一定合理,从而耽误后续计算。因此,在这一节,我们将介绍一种结构优化策略——\emph{分步优化}。通过逐步提高精度,快速找到体系的合理结构。

\subsection{使用Gamma点}\label{subsec:一些结构优化的策略与加速方法-使用Gamma点}

在优化周期性结构时,我们通常会使用一定大小的k点来提高计算精度。但对于初始而言,这种密度往往会造成不必要的时长,而且在初始结构不合理的时候,使用一个较小的K点也足以帮助我们找到一个大致正确的结构。此时一个较合理的方法是\emph{使用Gamma点}。

如果是使用vaspkit生成的\code{KPOINTS},只需使用\code{vaspkit-102-2-0}即可生成仅有Gamma点的\code{KPOINTS}。可以比较发现,使用这一方法可以较大提高计算速度。

\begin{extend}
    如果只使用Gamma点计算,在vasp计算版本中可以使用\code{vasp\_gam}版本计算。修改方法通常在提交任务脚本中。例如,以我所在课题组为例,对于105、106和224节点,只需要在已有提交脚本中找到\code{vasp\_std}并将其修改为\code{vasp\_gam}即可。
\end{extend}

\subsection{固定相互作用较弱的原子}\label{subsec:一些结构优化的策略与加速方法-固定相互作用较弱的原子}

对于某些体系,例如气体吸附,气体分子与靠近气体分子的一层原子的相互作用应当明显大于远离气体分子的那一层原子。因此,在进行低精度结构优化时,可以优先将相互作用较弱的原子进行固定,而只优化相互作用较强的原子,从而使其快速找到合适位置。

以气体吸附为例,在初始结构不合理的时候,可以将吸附衬底固定而只优化气体分子所在位置,从而将其移动到大致正确的位置,再进行后续高精度优化。

\subsection{使用低精度参数}\label{subsec:一些结构优化的策略与加速方法-使用低精度参数}

如果你所需要的优化精度(如\code{EDIFF},\code{EDIFFG}等)较高,可以优先使用低精度进行优化,再将其逐渐提升到高精度,从而使其快速收敛到合适位置。

在必要的时候,稍微调大\code{POTIM}参数从而增大迭代步长进行快速优化也是合理的,但其参数的调整需要人为“监控”以避免出现不合理的情况(例如,步长过大导致迭代不收敛)。

在调整\code{EDIFF}和\code{EDIFFG}时,如果\code{EDIFF}设置精度较低而\code{EDIFFG}设置精度较高,有可能会发生力难以收敛的情况。因此,在提高二者精度时,可以先提高\code{EDIFF}的精度,再提高\code{EDIFFG}的精度\footnote{这一部分只是个人经验之谈。}。

此外,在进行结构优化参数设置时,一些与结构可能无关的参数初始也可以不设置(例如,对于没有范德华作用体系的结构,考虑范德华作用则是多余的)。但是,\emph{磁性考虑与否可能会影响结构},因此,在进行分步优化时,应当考虑磁性(\code{ISPIN=2})以避免磁性对结构的影响。
