% 请在下方的大括号相应位置填写正确的节标题和标签,以及作者姓名
\section{对晶格常数进行优化\code{ISIF=3}}\label{sec:对晶格常数进行优化ISIF=3}
\sectionAuthor{Jiaqi Z.}

% 请在下方的item内填写本节知识点
\begin{Abstract}
    \item 如何对晶格常数进行优化?
    \item 如何固定晶格常数坐标?
\end{Abstract}

在本节,我们将进一步讨论结构优化,考虑非分子、团簇情况,即对\emph{晶体结构}的结构优化。与分子的情况不同,晶体结构需要考虑晶格常数的大小。在VASP当中,我们使用\code{ISIF=3}表示对晶格常数的优化。

首先,我们会展示一个简单的例子——对晶体硅进行结构优化,接着,我们将以二维材料为例,讨论如何对低维材料(如二维材料、一维材料)进行结构优化。其中,后者由于\emph{真空层}的设置,在优化时可能需要一些其他设置。

\subsection{三维材料结构优化}\label{subsec:对晶格常数进行优化ISIF=3-三维材料结构优化}

首先,我们需要构建Si的晶体结构。借助于Materials Project网站,可以下载到对应晶体的cif文件,并使用如VESTA等软件将其导出至\code{POSCAR}文件。作为练习,你可以参考下面的文件创建你的\code{POSCAR}文件:

\begin{lstlisting}[caption=POSCAR]
Si2
1.0
        3.8669745922         0.0000000000         0.0000000000
        1.9334872961         3.3488982326         0.0000000000
        1.9334872961         1.1162994109         3.1573715331
    Si
    2
Direct
        0.750000000         0.750000000         0.750000000
        0.500000000         0.500000000         0.500000000
\end{lstlisting}

可以发现,晶体Si为三方晶系(满足$a=b=c,\alpha=\beta=\gamma\ne90^\circ$)。与优化分子不同,晶体需要同时进行原子坐标优化和\emph{晶格常数}优化。与优化原子坐标类似,如果需要优化晶格常数,需要设置\code{ISIF=3}(其他参数可以参考\ref{sec:对坐标进行优化ISIF=2}一节中的\code{INCAR}设置。下面给了一个可以直接使用的\code{INCAR}文件:

\begin{lstlisting}[caption=INCAR]
ENCUT  =  600
LWAVE  = .FALSE.
LCHARG = .FALSE.
PREC   = Accurate
NSW    =  300
ISMEAR =  0
SIGMA  =  0.05
IBRION =  2
ISIF   =  3
EDIFFG = -0.01
\end{lstlisting}

可以直接使用\code{vaspkit-101-LR}生成。

\begin{attention}
    也可以使用\code{vaspkit-101-SR}生成\code{INCAR}文件,但需要将\code{ISIF}设置为3(vaspkit在这一选项下默认设置为2)
\end{attention}

与分子不同,对于晶格我们需要设置更多的k点(体现在\code{KPOINTS}文件中),通常在一个方向上的k点个数与晶格常数近似为$ka\sim20$即可。而在实际使用时,我们完全可以借助于vaspkit脚本,使用\code{vaspkit-102-2}生成以Gamma为中心的K点,输入需要使用的K点密度即可生成对应的\code{KPOINTS}文件(同时使用默认配置生成\code{POTCAR})。我们使用0.06生成的\code{KPOINTS}如下:

\begin{lstlisting}[caption=KPOINTS]
K-Spacing Value to Generate K-Mesh: 0.060
0
Gamma
   5   5   5
0.0  0.0  0.0
\end{lstlisting}

提交上述任务,完成后输出的\code{CONTCAR}文件为:

\begin{lstlisting}[caption=CONTCAR]
Si2                                     
1.00000000000000     
    3.8715704308906274   -0.0000000000000000    0.0000000000000000
    1.9357852154453137    3.3528783456576567    0.0000000000000420
    1.9357852154453137    1.1176261152526314    3.1611240196775290
Si
    2
Direct
0.7500000000000000  0.7500000000000000  0.7500000000000000
0.5000000000000000  0.5000000000000000  0.5000000000000000

0.00000000E+00  0.00000000E+00  0.00000000E+00
0.00000000E+00  0.00000000E+00  0.00000000E+00
\end{lstlisting}

可以发现,晶格常数相较于初始\code{POSCAR}发生了变化。同时,Si-Si键长从2.368 Å变为2.371 Å.

\begin{attention}
    从原子坐标上看,似乎原子位置没有优化,但由于我们使用的是\emph{分数坐标},对于同样的原子坐标,不同晶格常数对应不同的键长与原子间相对位置。
\end{attention}

\subsubsection{*关于vaspkit的k点设置}

\begin{extend}
    无论你是计算新手还是有一定经验的“老手”,在使用vaspkit时可能产生过疑问:\emph{设置k点时输入的密度(例如0.06)到底是什么}?上面的示例中为什么生成的k点个数就是$5\times5\times5$?这一部分我们试图回答这一问题。

    首先,k点所针对的是倒空间,而我们所使用的\code{POSCAR}都是针对于\emph{实空间}的。通过实空间的晶格基矢,我们可以求得倒空间的基矢。例如,倒空间基矢$\vb{b}_1$就可以用下面的方式计算:

    \begin{equation*}
        \vb{b}_1 = 2\pi \frac{\vb{\alpha}_2\times\vb{\alpha}_3}{\vb{\alpha}_1\cdot(\vb{\alpha}_2\times\vb{\alpha}_3)}
    \end{equation*}

    因此,我们可以通过实空间的晶格基矢长度,计算得到倒空间的基矢长度。例如,在本例中,基矢长度就是1.99. 

    而我们输入的密度$n$,则可以定义为$n=b/2\pi k$,其中$b$为倒空间基矢的长度。例如,在本例中,倒空间基矢长度为1.99,则可以带入得到当$n=0.06$时,可得$k\approx5$.

    或者,可以将密度解释为\emph{在$b/2\pi$中平均撒入$k$个点,相邻两点的距离。}
\end{extend}

\subsection{固定某一晶格常数的优化}\label{subsec:对晶格常数进行优化ISIF=3-固定某一晶格常数的优化}

对于低维材料(本节后续均以二维材料举例),为了避免层与层之间的相互作用,在构建结构时需要在垂直于平面方向的晶格常数设置较大的值(真空层)。以二维\ch{TiS2}为例,在$z$方向设置长度为15 Å的真空层,构建得到的\code{POSCAR}如下所示:

\begin{lstlisting}[caption=POSCAR]
"Ti1 S2"
1.0
        3.4143800735         0.0000000000         0.0000000000
        -1.7071904305         2.9569396545         0.0000000000
        0.0000000000         0.0000000000         15.000000000
    Ti    S
    1    2
Direct
        0.000000000         0.000000000         0.500000000
        0.666670024         0.333330005         0.579720020
        0.333330005         0.666670024         0.420280010
\end{lstlisting}

如果不加以限制,参考前面关于三维材料的结构优化所使用的设置,可以很容易想到(事实也会如此):\emph{真空层会发生变化}。这在物理上没有实际意义,因此需要想办法对真空层($c$轴)进行固定。

\subsubsection{使用\code{OPTCELL}文件进行固定}

\begin{extend}
    使用\code{OPTCELL}之前需要对VASP进行再编译,特别是需要重写文件\code{constr\_cell\_relax.F}。具体的编译原理请参考刘锦程博士的博客\footnote{https://blog.shishiruqi.com//2019/05/05/constr/},在此不再赘述。

    如果无法配置,请联系你所在课题组的服务器运维人员。
\end{extend}

在使用\code{OPTCELL}进行固定时,需要创建一个名为\code{OPTCELL}的文件,内部为$3\times3$的矩阵,例如,在本例中我们希望对真空层进行固定,则所创建的文件如下所示:

\begin{lstlisting}[caption=OPTCELL]
100
110
000
\end{lstlisting}

其中,\emph{1表示对某个坐标分量进行优化,而0表示固定}(不优化)。

\begin{attention}
    在\code{OPTCELL}当中,所有数字之间没有空格,这与下面要介绍的\code{IOPTCELL}不同。
\end{attention}

\subsubsection{使用\code{IOPTCELL}进行固定}

肖承诚博士在github上提供了一个补丁\footnote{https://github.com/Chengcheng-Xiao/VASP\_OPT\_AXIS},可以通过在\code{INCAR}中设置\code{IOPTCELL}参数实现对晶格的固定。其基本原理是对通过构建力矩阵实现固定,假设原始的晶格常数构成一个$3\times3$的矩阵$A_\text{old}$,\code{IOPTCELL}的作用就是设置一个$3\times3$的力矩阵$F$,使得优化后得到的新晶格常数$A_\text{new}=A_\text{old}\times F$在需要固定的方向上元素值为0.

还是以上面的\ch{TiS2}为例,当设置力矩阵$F$为
\begin{equation*}
    F=\mqty(1&0&0\\1&1&0\\0&0&0)
\end{equation*}
时,可以实现新的晶格常数矩阵为
\begin{equation*}
    A_\text{new}=\mqty(1&0&0\\1&1&0\\0&0&0)\cross\mqty(1&0&0\\2&1&0\\0&0&0)
\end{equation*}
从而实现对真空层的固定。

在设置时,在\code{INCAR}里面提供参数\code{IOPTCELL},按照\emph{列顺序}书写(先写第一列,再写第二列,最后写第三列)。一共9个数字\emph{写在同一行}且中间\emph{用空格分隔}。例如,本例则可以设置为\code{IOPTCELL=1 1 0 0 1 0 0 0 0}