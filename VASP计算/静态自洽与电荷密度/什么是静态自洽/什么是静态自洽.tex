% 请在下方的大括号相应位置填写正确的节标题和标签,以及作者姓名
\section{什么是静态自洽}\label{sec:什么是静态自洽}
\sectionAuthor{Jiaqi Z.}

\begin{Abstract}
    \item 什么是静态自洽
    \item 为什么要进行静态自洽
\end{Abstract}

在本章,我们将开始讨论电子性质的计算。电子性质的计算范围非常大,小到体系的能量,大到计算体系的能带、态密度等都是对材料电子性质进行计算。一些复杂的计算如能带、态密度等我们放到后面的章节单独介绍,而一些简单的性质将在本章进行说明。

\subsection{什么是静态自洽}\label{subsec:什么是静态自洽-什么是静态自洽}

在第\ref{chap:结构优化}章中介绍过,在进行任何材料的计算前,都需要进行\emph{结构优化}。结构优化是VASP计算的第一步,旨在通过调整输入体系的坐标,得到一个相对稳定的基态结构。这个过程包括\emph{原子迟豫}和\emph{电子迭代}两个嵌套的过程。每次计算中都进行原子迟豫和电子迭代计算,直到达到自动中断或者最大原子迟豫步数。

相比之下,静态自洽计算(self-consistent field method, SCF)是在结构优化的基础上进行的。此时,\emph{原子位置保持不动,只对电子进行自洽计算,以达到体系的最低能量}。静态自洽计算前需要使用结构优化输出的\code{CONTCAR}文件,并将其拷贝成新的POSCAR文件。

\begin{extend}
    在VASP计算中(甚至包括其他大多数材料计算软件),自洽方法的基本思想都是先按照某种方法给出波函数的一个估计,然后利用这个估计来计算电子密度,再通过电子密度来得到哈密顿量中与粒子间相互作用有关的项,再进行薛定谔方程的求解得到一组改进的估计。
\end{extend}

\subsection{为什么要进行静态自洽计算}\label{subsec:什么是静态自洽-为什么要进行静态自洽计算}

计算静态自洽的本质是为了得到体系的电子性质,在以下几种情况下,你可能会需要计算静态自洽:

\begin{itemize}
    \item 计算能量、磁矩等性质:要得到体系的能量,首先需要得到能量最低时的电子密度(这是“密度泛函理论”的基本假设);虽然在计算结构优化时也可以得到一个能量,但有时结构优化为了提高计算效率,可能会忽略一些与结构影响无关的近似(如范德华校正等),但这些近似会影响计算的能量值。在实际计算时,我们可能会在自洽计算过程中考虑更多的近似;
    \item 计算态密度、能带等电子性质:其本质上也是为了计算得到电子结构性质;
    \item 计算电荷密度图等性质:这种直接计算电荷密度的性质,也必然要进行自洽计算。
\end{itemize}

在本章,我们主要讨论能量、磁矩、电荷密度(特别是常见的“差分电荷密度”)的计算方法。它们的计算大多只需要一些简单的物理知识即可说明清楚。一些更深入的电子性质(如bader电荷转移、电子局域函数ELF、晶体轨道布居COHP等)将在后面的章节进行讨论,它们有些需要更复杂的物理知识,甚至可能需要一些更复杂的计算(借助其他程序包)。